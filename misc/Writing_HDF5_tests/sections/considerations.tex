\documentclass[../HDF5_RFC.tex]{subfiles}
 
\begin{document}

\section{General considerations}
\label{considerations}

The following are general considerations that test authors should keep in mind when writing HDF5 tests.

\subsection{Coupling of test programs and sub-tests}

It is best to avoid any coupling between test programs, as well as sub-tests in a test program, where possible. This is often a problem when one test depends on a previous test having created a file. When
tests are coupled in this manner and an earlier test fails, it can be difficult to diagnose the problems
in the tests that run later on in the sequence of tests. It also reduces testing coverage until issues
in the earlier tests/sub-tests are fixed and adds complexity to the build system code (where dependencies
between tests have to be explicitly added). A good practice is to make use of helper functions to factor
out duplicated code between tests and sub-tests to avoid this coupling.

\subsection{Suppress HDF5 error stacks only as needed}

By default, HDF5 will print out a stack of error messages to \texttt{stderr} when an API function fails.
Before a refactoring, the 'testframe' framework previously would disable HDF5's error stack mechanism
and rely on tests to explicitly print out the HDF5 error stack as needed. This adds maintenance overhead
for test authors and also makes debugging of tests more difficult. Test programs should generally leave
HDF5's error stacks enabled (unless disabled explicitly via a command-line argument or some other
mechanism) and add error stack suppression calls around operations that are expected to fail and cause
error stack output. Such operations can be surrounded by the \texttt{H5E\_BEGIN\_TRY} and
\texttt{H5E\_END\_TRY} macros, as in:

\begin{minted}{C}
H5E_BEGIN_TRY
{
    status = ...;
}
H5E_END_TRY
if (status < 0)
    error;
\end{minted}

Tests should capture some form of return value from the operation inside the macros and then check the
return value outside of the macros to avoid skipping over the logic in the \texttt{H5E\_END\_TRY} macro.
Otherwise, the HDF5 error stack may become disabled for the rest of the test program's execution,
potentially hiding error stacks from other test failures later on.

\subsection{Limit testing time by TestExpress value}

HDF5's testing framework functionality includes an internal variable that a test program can query the
value of and use to determine whether it should skip some portion of testing in order to complete faster. While only a few of HDF5's tests currently pay attention to this variable's value, it is especially
important for parallel and multi-threaded tests to use this value in conjunction with an application
timer in order to safeguard against deadlocks / livelocks and other such issues that can cause tests
to run for longer than intended. While CMake can kill off misbehaving tests after a certain amount of
time has passed, this timeout value (\texttt{DART\_TESTING\_TIMEOUT}) can be changed\footnote{Note that
it may also \textbf{need} to be changed, as the default value is a 20 minute timeout for a test and, as
noted further on, it may be desirable to allow 30 minutes for a test} (and sometimes is, for certain HPC machines), allowing misbehaving parallel/multi-threaded tests to consume more testing time unnecessarily. Further, Autotools (which doesn't implement this functionality) will need to continue being supported for
the foreseeable future. The value of this variable can be retrieved by calling either
\nameref{apdx:testframe_gettestexpress} (for 'testframe' tests) or \nameref{apdx:h5test_h5gettestexpress}
(for 'h5test' tests). For 'testframe' tests, the value of this variable can be changed at runtime
by passing one of the below values to the \texttt{-testexpress} command-line option.

The meanings for the different values of the \texttt{TestExpress} variable are as follows:

\begin{itemize}

    \item 0 - Tests should take as long as necessary
    \item 1 - Tests should take no more than 20 minutes
    \item 2 - Tests should take no more than 10 minutes
    \item 3 (or higher) - Tests should take no more than 1 minute (default in HDF5)

\end{itemize}

Note that the limit imposed by the value of the \texttt{TestExpress} variable is intended to be the limit
on the total runtime of an individual test program, even if that test program consists of multiple
sub-tests. This implies that test programs should:

\begin{itemize}

    \item Set an \texttt{alarm(2)}-like timer at the start of the test program to ensure that the test
          program exits in a timely fashion according to the value of \texttt{TextExpress}. Note that
          HDF5's testing framework has the \nameref{apdx:testframe_testalarmon} / \nameref{apdx:testframe_testalarmoff} functions, but these currently use a compile-time value
          for the timer that can only be overridden by an environment variable
          (\texttt{HDF5\_ALARM\_SECONDS}). They also depend on the availability of the \texttt{alarm(2)} function, which shouldn't generally be an issue at this point but may be problematic for MinGW builds as commented on in \texttt{testframe.c}. Some more discussion on portable timers for
          testing is discussed in section \ref{sec:testing_timeouts}.
    \item Keep track of the number of sub-tests that will be performed and divide the total allowed
          runtime among sub-tests to ensure that they all run to some degree of completion, even for more
          restrictive values of \texttt{TestExpress}. Note that some margin may need to be included in
          this calculation so that each sub-test can run to some degree of completion (while possibly
          spilling over their timeout by a small amount) and the test program can cleanup without running
          into the test alarm.
    \item Engineer sub-tests to accept and gracefully handle specified timeout values, taking care to check
          against the timeout value relatively frequently so as to not steal runtime from other sub-tests.
          Setting some form of flag when a sub-test or the entire test program has exceeded an allowed
          timeout value would also be a reasonable approach to give time for test cleanup. Alternatively, POSIX timers could be used to simplify test logic a bit.
    \item Engineer parallel/multi-threaded tests to adjust workload to try to accommodate the
          \texttt{TestExpress} runtime limit as the number of processes/threads involved increases

\end{itemize}

While some of the above items may not be applicable to tests in general, authors of parallel and
multi-threaded test programs should be especially mindful of how long their test program runs and
should be aware of the potential for hangs during execution of their test program.

\subsection{Be VOL connector and VFD aware}

Test authors should be aware that their tests may end up being run with a VOL connector or Virtual File
Driver that is different from the default. This can result in test failures when testing behavior that
is specific to the native HDF5 VOL connector and/or default sec2 VFD. Tests should make use of the
\nameref{apdx:h5test_h5usingnativevol} utility function from \texttt{h5test.c}, as well as any capability
flags queried from a VOL connector with \href{https://support.hdfgroup.org/documentation/hdf5/latest/group___f_a_p_l.html#ga2ad4dc5c6e5e4271334a7b1c6ee0777d}{\texttt{H5Pget\_vol\_cap\_flags()}}, in order to skip testing of functionality that non-native VOL connectors may not support. Tests should make use of the various \texttt{h5\_XXX\_driver\_XXX()} utility functions from \texttt{h5test.c} in order to skip testing of 
functionality that non-sec2 file drivers may not support. Tests should make use of the \nameref{apdx:h5test_h5fixname} utility function from \texttt{h5test.c} in order to generate file names that
make sense for different file drivers.

\subsection{Make use of 'testframe' verbosity levels}

For tests that integrate with the 'testframe' testing framework, test authors are encouraged to make use
of the framework's verbosity level settings. This allows tests to include printing out of debugging
information and other useful output without needing to recompile the test. When a problem occurs, the test
can then be debugged by simply passing a specific value to the \texttt{-verbose} command-line option,
making the debugging process simpler and preventing debug code from going stale.

\subsection{Temporary files}

Test programs should offer the ability to skip deleting of temporary files that were created by the test
when it exits. It is very useful for debugging purposes to have those temporary files around when a test
is failing. For 'testframe' tests, this can be achieved by checking the return value of the
\nameref{apdx:testframe_gettestcleanup} function before calling \href{https://support.hdfgroup.org/documentation/hdf5/latest/group___h5_f.html#ga2e8b5e19b343123e8ab21442f9169a62}{\texttt{H5Fdelete()}}
or similar. For 'h5test' tests, this can be achieved by making use of the \nameref{apdx:h5test_h5cleanup} function to cleanup test files. Both functions are affected by whether the \texttt{HDF5\_NOCLEANUP} environment variable is defined. If it is defined, \nameref{apdx:testframe_gettestcleanup} will return \texttt{false} and \nameref{apdx:h5test_h5cleanup} will avoid cleaning up test files, as long as the test program previously called one of the \texttt{h5\_fixname(\_xxx)} functions.

\subsection{Test initialization functions}

HDF5 tests that integrate with the 'testframe' testing framework should make sure to call the function
\nameref{apdx:testframe_testinit} as soon as possible after HDF5 is initialized. Tests that don't integrate with the 'testframe' testing framework should make sure to call the function \nameref{apdx:h5test_h5testinit} as early as possible in the \texttt{main()} function. These functions set up testing infrastructure state
such as the \texttt{TestExpress} functionality. For serial tests, these calls should generally come before
any use of HDF5. For parallel tests, these calls should come before any use of HDF5, but also \textbf{after}
a call to \texttt{MPI\_Init()}.

\subsection{Flush output stream before aborting testing}

If an HDF5 test uses a function, such as \texttt{abort()} or \texttt{MPI\_Abort()}, to abruptly terminate program execution on errors rather than gracefully handling them, it should be sure to flush the output
stream (e.g., \texttt{stderr} or \texttt{stdout}) used to ensure the output isn't lost.

\subsection{Scope of functions and global variables}

All functions and global variables in a test should be marked as \texttt{static} unless there is reason
for them to have wider scope. This prevents some oddities that can occur when different tests have
similarly-named functions or variables and is a best practice in general. If a test author is considering
a wider scope for a function or variable for use in another test, it is likely that the function or
variable should simply be moved into one of the testing frameworks as appropriate.

\end{document}