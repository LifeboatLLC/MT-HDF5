\documentclass[../HDF5_RFC.tex]{subfiles}
 
\begin{document}

\section{Further considerations}
\label{further_considerations}

\subsection{Parallel tests}

\subsubsection{MPI implementations}

After a parallel HDF5 test program is reasonably complete, it is always a good idea to run that test
program against multiple MPI implementations to catch problems that may occur from differences in
implemented functionality, values of limits, etc. between the MPI implementations. For parallel HDF5,
this usually means a test program should, at a minimum, be tested against recent versions of
\href{https://www.mpich.org/}{MPICH} and \href{https://www.open-mpi.org/}{OpenMPI}.

\subsection{Multi-threaded tests}

\subsubsection{\texttt{TestExpress} values}

The \texttt{TextExpress} variable is currently set to a default value of 3 in order to facilitate quicker
CI testing of HDF5. While this chosen value may be acceptable for testing multi-threaded HDF5 functionality
for changes in GitHub PRs, this value will be too restrictive for general testing of multi-threaded HDF5.
For extensive manual testing of multi-threaded 'testframe' tests, the \texttt{TextExpress} value can be overridden at runtime with the \texttt{-testexpress} command-line argument, or HDF5 can be configured with
a different default level. If it is desired that multi-threaded HDF5 functionality be tested against changes
in GitHub PRs, multi-threaded tests could be split into a separate GitHub CI action to minimize their
effects by running them in parallel alongside other testing. However, this would likely still impose a
30 minute time limit as the general consensus seems to be that tests which take longer than this tend to
slow down the development process when PRs are actively being created. More exhaustive testing will likely
need to occur outside of GitHub, as actions are limited to 6 hours of runtime unless being executed on a machine not owned by GitHub. An effective cadence for exhaustive testing would need to be determined so
that developers can be informed reasonably quickly when new issues are introduced to multi-threaded HDF5.

\subsubsection{Facilitating debugging of multi-thread issues}

When debugging an issue that occurs while testing multi-threaded HDF5, it is advantageous to stop the
process as immediately as possible in order to prevent program state from progressing. To this end, it
has been proposed that \texttt{assert()} statements should be used to achieve this effect while running
from within a debugger. To be the most effective, this would likely require that \texttt{assert()}
statements be included in both the HDF5 testing framework and the HDF5 library itself. While the former
should be fairly easy to accomplish with effort from a test program's author and also by updating some
of the testing macros in \texttt{h5test.h}, the latter is of more concern for general use of HDF5. While \texttt{assert()} statements will only have an effect in debug builds of HDF5, past experience has shown
that users employ these types of builds for general use. It is proposed that the "developer" build mode\footnote{\href{https://github.com/HDFGroup/hdf5/pull/1659}{HDF5 PR \#1659}}, which was added to
HDF5's CMake build code, should also be added to the library's Autotools build code (provided that it
remains around for the foreseeable future) and these \texttt{assert()} statements should only be active
with a "developer" build of the library. This allows for much more freedom to introduce heavy-handed
debugging techniques, while not affecting users of HDF5 in general.

Also note that, after some research on the topic\footnote{\href{https://nullprogram.com/blog/2022/06/26/}{Assertions should be more debugger-oriented}}, it appears that whether or not an \texttt{assert()} statement stops at a useful place during debugging, or even stops at all, varies among platforms. However, compiler features can help out a bit here, with the following given as an example that might be useful:

\begin{minted}{C}
#ifdef DEBUG
#  if __GNUC__
#    define assert(c) if (!(c)) __builtin_trap()
#  elif _MSC_VER
#    define assert(c) if (!(c)) __debugbreak()
#  else
#    define assert(c) if (!(c)) *(volatile int *)0 = 0
#  endif
#else
#  define assert(c)
#endif
\end{minted}

These compiler directives are better suited for debugging purposes as they generate trap-like instructions and
generally force a much more abrupt process termination as desired. Testing across the compilers/platforms that
HDF5 currently supports would be needed to determine if this is a workable approach.

\subsubsection{Testing}

\begin{enumerate}

    \item In addition to running multi-threaded HDF5 tests regularly, it may also be useful to setup CI
          integration for testing HDF5 with one or both of the following tools:

          \begin{itemize}

              \item ThreadSanitizer - Tool from LLVM/Clang to detect data races. Could be useful with
                    some tweaking, but it's currently in beta and known to have issues / produce false
                    positives with the current approach to adapting HDF5 for multi-threaded usage.

              \item Helgrind - Valgrind tool for detecting several different types of errors in usage
                    of POSIX pthreads. Has not yet been tested with multi-threaded HDF5. Generally only
                    useful for POSIX pthreads, but can potentially be extended to other threading packages
                    with its code annotation capabilities.

          \end{itemize}

    \item An interesting related project\footnote{\href{http://www.cs.umd.edu/projects/PL/multithreadedtc/overview.html}{MultithreadedTC}} explores forcing specific thread interleaving, which may be useful in very specific testing circumstances

\end{enumerate}

\subsection{Testing timeouts}
\label{sec:testing_timeouts}

Currently, HDF5 handles timeouts in tests with immediate process termination, either through the
build system itself (in the case of CMake builds of the library), through process timers in
specific tests (in the case of Autotools builds of the library), or through higher level constructs
(in the case of testing on a specific system). As more threaded and parallel tests are added to HDF5,
the ability to gracefully handle testing timeouts becomes much more desirable. Ideally, HDF5 test
programs would be able to handle testing timeouts in such a manner that all sub-tests and cleanup
can be performed within a given amount of runtime. While no decision on an approach for revising
the handling of testing timeouts has been made yet, this section details how timeouts are currently
handled and some alternate approaches that were investigated.

\subsubsection{Timeouts in HPC job queues}

Toward the top in the hierarchy of testing timeout constructs are the job time limits imposed when
testing HDF5 on an HPC system\footnote{Note that this type of timeout can also apply on other systems in
general.}. HDF5 is usually tested in "debug" (or similar) job queues on HPC systems. These job queues
typically have relatively short limits ($\sim$30 minutes) on how long a job can run before it is killed.
As HDF5 developers usually don't have any control over this limit (outside of using a different job
queue), it should be considered a hard limit, which is why implementing the ability for tests to
adjust to the \texttt{TestExpress} functionality level is important.

\subsubsection{Timeouts in CMake builds}

Next in the hierarchy are testing timeouts imposed by the build system, which is currently only
applicable to CMake builds of HDF5. When HDF5 is built with CMake, testing is typically done with
CMake's test driver program, \href{https://cmake.org/cmake/help/latest/manual/ctest.1.html}{ctest}.
Tests that are run in this manner will be subject to a timeout value that can be controlled in
various ways by HDF5's CMake build logic. When the timeout value expires, the test will, currently,
be immediately terminated. By default, HDF5 sets a 20 minute timeout for all tests run through ctest.
While this value can be adjusted in several ways, doing so is uncommon and there does not appear to be
a useful, programmatic way of retrieving the value. As the ideal goal is to have control over test
behavior when a timeout occurs, some specific HDF5 tests have been given an infinite timeout from
the perspective of CMake, allowing the test timeout to be controlled further down in the hierarchy
\footnote{Note that as of version 3.27, CMake has somewhat improved control over testing timeouts
with \href{https://cmake.org/cmake/help/latest/prop_test/TIMEOUT_SIGNAL_NAME.html}{TIMEOUT\_SIGNAL\_NAME},
but HDF5 still only requires version 3.18 of CMake as a minimum and only moves forward as needed}.
As of the time of writing, this only includes a few multi-threaded tests that may end up running well
past the default CMake timeout value for a full run. Note that if test program binaries from a CMake
build of HDF5 are run manually, they will not have a timeout imposed by CMake itself and will rely
on any timeouts coordinated by the below constructs.

\subsubsection{Timeouts in Autotools builds}

Autotools builds of HDF5 do not impose any timeouts from the build system itself, but rather rely
on the test programs to use some form of internal process timer to catch timeouts. These are discussed
in the following subsections.

\subsubsection{'Standard' timers}

The timers interacted with by the \texttt{alarm(2)} and \texttt{setitimer()} / \texttt{getitimer()}
functions are what this document will refer to as 'standard' timers. These allow for a single active
timer per process and deliver their timeout notifications via a signal. The portability story for
these timers is ideal for HDF5, as \texttt{alarm(2)} is standard C (and is currently used as a timer
by HDF5's tests) and \texttt{setitimer()} / \texttt{getitimer()} are part of POSIX and appear to be
widely supported. However, there are a few issues with using these types of timers:

\begin{itemize}
    \item The fact that these timers deliver their timeout notifications via a signal severely limits
          what can be done in reaction to a test timeout, as many C/POSIX functions are not safe to
          call from within a signal handler. Further, the signal could be delivered to any thread in
          a multi-threaded test program and may complicate test timeout coordination among threads.
    \item The fact that only one active timer is allowed per process presents some difficulties or
          awkward designs when attempting to have tiered timeouts during testing. The thought here is
          that it would be ideal for HDF5 tests to have two different timers, one that controls the
          timeout for the entire allowed runtime of a test program and a second timer that controls
          the timeout for each sub-test, as some fraction of the entire allowed runtime of a test
          program.
\end{itemize}

As a first step, it has been proposed that a solution which uses these timers, a background thread and
a flag should be investigated. This would likely involve having the background thread wait until the
timeout is expired, at which point it will set a flag to notify the test program that it is time to
cleanup. Meanwhile, the test program will need to regularly check this flag in some manner. This solution
would likely only cover the use case of a timeout for the whole test program at first, but may be able
to be revised to cover per-subtest timeouts as well.

\subsubsection{POSIX realtime extension timers}

The timers interacted with by the \texttt{timer\_create()}, \texttt{timer\_gettime()},
\texttt{timer\_settime()}, etc. functions are what this document will refer to as 'POSIX RT' timers.
These allow for multiple timers per process and have the option of delivering their timeout notifications
via either a signal or a thread. While an improvement over the 'standard' timers, there are also a few
issues with using these types of timers:

\begin{itemize}
    \item Initial research appears to show that these timers are not portable to Mac OS or some (most? all?)
          of the BSDs, which HDF5 is routinely used on.
    \item Use of these timers requires linking in \texttt{librt} which may also necessitate bringing in
          the pthreads library and implicitly loading it at runtime as well, which may not be desirable
          for several reasons. This wouldn't generally be an issue for multi-thread or thread-safe
          builds of the library, but could be problematic for regular builds.
\end{itemize}

\subsubsection{Platform-specific timers}

The timers interacted with by the \texttt{timerfd\_create()} (Linux), \texttt{dispatch\_source\_create()} (Mac OS), \texttt{CreateTimer()} (Windows), etc. functions are what this document will refer to as 'platform-specific timers'. Initial research appears to show that most or all of the major platforms that HDF5 is
primarily supported on have platform-specific solutions which allow multiple active timers per process
which can deliver their timeout notifications in an acceptable manner for multi-threaded programs. For HDF5
test programs to make use of these timers, a framework of wrapper functions around the platform-specific functionality would have to be created. While this approach seems to be ideal for the future, the estimated
effort of implementing it appears large enough that it has not been proposed at this time.

\end{document}