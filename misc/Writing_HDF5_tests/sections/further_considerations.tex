\documentclass[../HDF5_RFC.tex]{subfiles}
 
\begin{document}

\section{Further considerations}
\label{further_considerations}

\subsection{Parallel tests}

\subsubsection{MPI implementations}

After a parallel HDF5 test program is reasonably complete, it is always a good idea to run that test
program against multiple MPI implementations to catch problems that may occur from differences in
implemented functionality, values of limits, etc. between the MPI implementations. For parallel HDF5,
this usually means a test program should, at a minimum, be tested against recent versions of
\href{https://www.mpich.org/}{MPICH} and \href{https://www.open-mpi.org/}{OpenMPI}.

\subsection{Multi-threaded tests}

\subsubsection{\texttt{TestExpress} values}

The \texttt{TextExpress} variable is currently set to a default value of 3 in order to facilitate quicker
CI testing of HDF5. While this chosen value may be acceptable for testing multi-threaded HDF5 functionality
for changes in GitHub PRs, this value will be too restrictive for general testing of multi-threaded HDF5.
If it is desired that multi-threaded HDF5 functionality be tested against changes in GitHub PRs,
multi-threaded tests could be split into a separate GitHub CI action to minimize their effects by running
them in parallel alongside other testing. However, this would likely still impose a 30 minute time limit
as the general consensus seems to be that tests which take longer than this tend to slow down the development
process when PRs are actively being created. More exhaustive testing will likely need to occur outside of
GitHub, as actions are limited to 6 hours of runtime unless being executed on a machine not owned by GitHub.
An effective cadence for exhaustive testing would need to be determined so that developers can be informed
reasonably quickly when new issues are introduced to multi-threaded HDF5.

\subsubsection{Facilitating debugging of multi-thread issues}

When debugging an issue that occurs while testing multi-threaded HDF5, it is advantageous to stop the
process as immediately as possible in order to prevent program state from progressing. To this end, it
has been proposed that \texttt{assert()} statements should be used to achieve this effect while running
from within a debugger. To be the most effective, this would likely require that \texttt{assert()}
statements be included in both the HDF5 testing framework and the HDF5 library itself. While the former
should be fairly easy to accomplish with effort from a test program's author and also by updating some
of the testing macros in \texttt{h5test.h}, the latter is of more concern for general use of HDF5. While \texttt{assert()} statements will only have an effect in debug builds of HDF5, past experience has shown
that users employ these types of builds for general use. It is proposed that the "developer" build mode\footnote{\href{https://github.com/HDFGroup/hdf5/pull/1659}{HDF5 PR \#1659}}, which was added to
HDF5's CMake build code, should also be added to the library's Autotools build code (provided that it
remains around for the foreseeable future) and these \texttt{assert()} statements should only be active
with a "developer" build of the library. This allows for much more freedom to introduce heavy-handed
debugging techniques, while not affecting users of HDF5 in general.

Also note that, after some research on the topic\footnote{\href{https://nullprogram.com/blog/2022/06/26/}{Assertions should be more debugger-oriented}}, it appears that whether or not an \texttt{assert()} statement stops at a useful place during debugging, or even stops at all, varies among platforms. However, compiler features can help out a bit here, with the following given as an example that might be useful:

\begin{minted}{C}
#ifdef DEBUG
#  if __GNUC__
#    define assert(c) if (!(c)) __builtin_trap()
#  elif _MSC_VER
#    define assert(c) if (!(c)) __debugbreak()
#  else
#    define assert(c) if (!(c)) *(volatile int *)0 = 0
#  endif
#else
#  define assert(c)
#endif
\end{minted}

These compiler directives are better suited for debugging purposes as they generate trap-like instructions and
generally force a much more abrupt process termination as desired. Testing across the compilers/platforms that
HDF5 currently supports would be needed to determine if this is a workable approach.

\subsubsection{Testing}

\begin{enumerate}

    \item In addition to running multi-threaded HDF5 tests regularly, it may also be useful to setup CI
          integration for testing HDF5 with one or both of the following tools:

          \begin{itemize}

              \item ThreadSanitizer - Tool from LLVM/Clang to detect data races. Could be useful with
                    some tweaking, but it's currently in beta and known to have issues / produce false
                    positives with the current approach to adapting HDF5 for multi-threaded usage.

              \item Helgrind - Valgrind tool for detecting several different types of errors in usage
                    of POSIX pthreads. Has not yet been tested with multi-threaded HDF5. Generally only
                    useful for POSIX pthreads, but can potentially be extended to other threading packages
                    with its code annotation capabilities.

          \end{itemize}

    \item An interesting related project\footnote{\href{http://www.cs.umd.edu/projects/PL/multithreadedtc/overview.html}{MultithreadedTC}} explores forcing specific thread interleaving, which may be useful in very specific testing circumstances

\end{enumerate}

\end{document}