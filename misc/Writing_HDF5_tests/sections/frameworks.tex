\documentclass[../HDF5_RFC.tex]{subfiles}

\begin{document}

\section{Testing frameworks}
\label{frameworks}

The following sections include details about the various testing frameworks in HDF5 and when to use
each one.

\subsection{'Testframe' framework}

HDF5 includes a testing framework contained within the \texttt{test/testframe.h} and \texttt{test/testframe.c}
files that has basic functionality such as skipping tests, controlling the amount of output from tests,
limiting the amount of time that tests run for and more (refer to Appendix \ref{apdx:test_frame} for
an overview of the framework). While the functionality from this testing framework is used for a handful
of older HDF5 tests, many new HDF5 tests end up being written as a simple \texttt{main()} function
that runs tests directly. However, it is recommended that test authors integrate their tests with the
'testframe' framework going forward, if possible. The ability to prevent tests from running too long is
especially important for parallel and multi-threaded tests, where forgetting to enable a test timer can
mean that a test runs for a very long time if it hangs. This is also especially important for HPC
environments, where the job queues used typically have a relatively short maximum running time (often around
30 minutes - 1 hour) and a hanging test will prevent getting results from other tests as they won't have
a chance to run. For the time being, the \texttt{testframe.h} header file also includes \texttt{h5test.h}, so the entire serial testing framework infrastructure is available to tests that integrate with the 'testframe' framework.

Recent work has shown that integrating with this testing framework can have some rough edges, especially due
to the fact that each test is added \textbf{before} the command-line arguments are parsed. However,
integration with this testing framework brings several benefits and is more maintainable than the current approach to writing new HDF5 tests completely from scratch. Refer to Appendix \ref{apdx:testframe_example}
and \ref{apdx:testframe_parallel_example} for examples of skeleton test programs integrating with this
testing framework.

\subsection{'H5test' framework}

HDF5 also includes a set of utility macros and functions contained within the \texttt{test/h5test.h} and \texttt{test/h5test.c} files for use by tests (refer to Appendix \ref{apdx:h5test} for an overview of
the framework). HDF5 tests should include the \texttt{h5test.h} header if they need access to the relevant utility macros and functions. HDF5 tests which don't integrate with the 'testframe' testing framework
typically use this framework, but 'testframe' tests may also make use of it.

\subsection{'Testpar' framework}

HDF5 also includes a set of utility macros and functions contained within the \texttt{testpar/testpar.h}
and \texttt{testpar/testpar.c} files for use by parallel tests (refer to Appendix \ref{apdx:testpar} for
an overview of the framework). In general, all parallel tests should include the \texttt{testpar.h} header. This header also includes the \texttt{h5test.h} and \texttt{testframe.h} headers, so the entirety of HDF5's testing framework infrastructure is currently available to parallel tests.

\end{document}