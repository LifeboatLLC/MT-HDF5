\documentclass[../HDF5_RFC.tex]{subfiles}
 
\begin{document}

% Start each appendix on a new page
\newpage

\section{Appendix: 'H5test' testing framework}
\label{apdx:h5test}

The following sections give an overview of HDF5's 'h5test' testing framework which is contained
within the \texttt{test/h5test.h} and \texttt{test/h5test.c} files. Details are only given for the
types, functions, etc. that appear in the relevant header files, while implementation details
in the relevant \texttt{.c} files are not generally given.

\subsection{Macros}

\begin{minted}{C}
#define TESTING(WHAT)
\end{minted}

Macro to print out the word "Testing" followed by the string specified in \texttt{WHAT}. As a side
effect, the global variable \texttt{n\_tests\_run\_g} is incremented. This macro does not print a
newline and should therefore always be followed by a matching usage of one of the following macros.

\begin{minted}{C}
#define PASSED()
#define H5_FAILED()
#define SKIPPED()
#define H5_WARNING()
#define TEST_ERROR
#define PUTS_ERROR(s)
#define STACK_ERROR
#define FAIL_STACK_ERROR
#define FAIL_PUTS_ERROR(s)
\end{minted}

Macros to give information output about a test's pass/fail status and reasons for failure, if
applicable.

\texttt{PASSED()} prints out " PASSED". As a side effect, the global variable
\texttt{n\_tests\_passed\_g} is incremented.

\texttt{H5\_FAILED()} prints out "*FAILED*". As a side effect, the global variable
\texttt{n\_tests\_failed\_g} is incremented.

\texttt{SKIPPED()} prints out " -SKIP-". As a side effect, the global variable
\texttt{n\_tests\_skipped\_g} is incremented.

\texttt{H5\_WARNING()} prints out "*WARNING*".

\texttt{TEST\_ERROR} prints out "*FAILED*", as well as the file name, line number and function name
where usage of the macro occurred and then uses \texttt{goto} to skip to the "error" label statement.
As a side effect, the global variable \texttt{n\_tests\_failed\_g} is incremented.

\texttt{PUTS\_ERROR(s)} prints out the string specified in \texttt{s}, as well as the file name, line
number and function name where usage of the macro occurred and then uses \texttt{goto} to skip to the
"error" label statement.

\texttt{STACK\_ERROR} prints out the contents of the default HDF5 error stack and then uses \texttt{goto}
to skip to the "error" label statement.

\texttt{FAIL\_STACK\_ERROR} is similar to \texttt{STACK\_ERROR}, but first prints out "*FAILED*", as well
as the file name, line number and function name where usage of the macro occurred. As a side effect, the global variable \texttt{n\_tests\_failed\_g} is incremented.

\texttt{FAIL\_PUTS\_ERROR(s)} is similar to \texttt{PUTS\_ERROR(s)}, but first prints out "*FAILED*", as
well as the file name, line number and function name where usage of the macro occurred. As a side effect,
the global variable \texttt{n\_tests\_failed\_g} is incremented.

\begin{minted}{C}
#define AT()
\end{minted}

Macro to print the file name, line number and function name where usage of the macro occurs.

\begin{minted}{C}
#define H5_FILEACCESS_VFD    0x01
#define H5_FILEACCESS_LIBVER 0x02
\end{minted}

Macros for values that can be passed to the \texttt{h5\_fileaccess\_flags()} function.

\begin{minted}{C}
#define H5_EXCLUDE_MULTIPART_DRIVERS     0x01
#define H5_EXCLUDE_NON_MULTIPART_DRIVERS 0x02
\end{minted}

Macros for values that can be passed to the \texttt{h5\_driver\_uses\_multiple\_files()} function.

\begin{minted}{C}
#define H5TEST_FILL_2D_HEAP_ARRAY(BUF, TYPE)
\end{minted}

Utility macro to fill a 2-dimensional array \texttt{BUF} with increasing values starting from 0.
\texttt{BUF} should point to a \texttt{struct \{ TYPE arr[...][...]; \}}.

\begin{minted}{C}
#define H5_TEST_EXPRESS_EXHAUSTIVE 0
#define H5_TEST_EXPRESS_FULL       1
#define H5_TEST_EXPRESS_QUICK      2
#define H5_TEST_EXPRESS_SMOKE_TEST 3
\end{minted}

Macros for the different TestExpress level settings.

\begin{minted}{C}
#define h5_free_const(mem) free((void *)(uintptr_t)mem)
\end{minted}

Utility macro that can be used to cast away \texttt{const} from a pointer when freeing it in
order to avoid compiler warnings.

\subsection{Enumerations}

None defined

\subsection{Structures}

None defined

\subsection{Global variables}

\begin{minted}{C}
uint64_t vol_cap_flags_g;
\end{minted}

Global variable to contain the capability flags for the current VOL connector in use, if applicable.
Typically set at the start of a test program with a call to \texttt{H5Pget\_vol\_cap\_flags()} and
primarily used to skip tests which use functionality that a VOL connector doesn't advertise support
for.

\begin{minted}{C}
MPI_Info h5_io_info_g;
\end{minted}

Global \texttt{MPI\_Info} variable used by some parallel tests for passing MPI hints to parallel
test operations.

\subsection{Functions}

\subsubsection{\texttt{h5\_test\_init()}}
\label{apdx:h5test_h5testinit}

\begin{minted}{C}
void
h5_test_init(void);
\end{minted}

Performs test framework initialization. Should be called by 'h5test' tests toward the beginning of the \texttt{main()} function after HDF5 has been initialized. Should \textbf{not} be called by 'testframe'
tests.

\subsubsection{\texttt{h5\_restore\_err()}}

\begin{minted}{C}
void
h5_restore_err(void);
\end{minted}

Restores HDF5's default error handling function after its temporary replacement by
\texttt{h5\_test\_init()}, which sets the error handling function to the \texttt{h5\_errors()}
callback function instead.

\subsubsection{\texttt{h5\_get\_testexpress()}}
\label{apdx:h5test_h5gettestexpress}

\begin{minted}{C}
int
h5_get_testexpress(void);
\end{minted}

Returns the current \texttt{TestExpress} level setting, which determines whether a test program should
expedite testing by skipping some tests. If the \texttt{TestExpress} level has not yet been set, it will
be initialized to the default value (currently level 1, unless overridden at configuration time). The
different \texttt{TestExpress} level settings have the following meanings:

\begin{itemize}

    \item 0 - Tests should take as long as necessary
    \item 1 - Tests should take no more than 30 minutes
    \item 2 - Tests should take no more than 10 minutes
    \item 3 (or higher) - Tests should take no more than 1 minute

\end{itemize}

If the \texttt{TestExpress} level setting is not yet initialized, this function will first set a local variable to the value of the \texttt{H5\_TEST\_EXPRESS\_LEVEL\_DEFAULT} macro, if it has been defined.
If the environment variable \texttt{HDF5TestExpress} is defined, its value will override the local
variable's value. Acceptable values for the environment variable are the strings "0", "1" and "2"; any
other string will cause the variable to be set to the value "3". Once the value for the local variable
has been determined, \texttt{h5\_get\_testexpress()} returns that value.

\subsubsection{\texttt{h5\_set\_testexpress()}}

\begin{minted}{C}
void
h5_set_testexpress(int new_val);
\end{minted}

Sets the current \texttt{TestExpress} level setting to the value specified by \texttt{new\_val}. If
\texttt{new\_val} is negative, the \texttt{TestExpress} level is set to the default value (currently
level 1, unless overridden at configuration time). If \texttt{new\_val} is greater than the highest
\texttt{TestExpress} level (3), it is set to the highest \texttt{TestExpress} level.

\subsubsection{\texttt{h5\_fileaccess()}}

\begin{minted}{C}
hid_t
h5_fileaccess(void);
\end{minted}

Wrapper function around \texttt{h5\_get\_vfd\_fapl()} and \texttt{h5\_get\_libver\_fapl()} which returns
a File Access Property List that has potentially been configured with a non-default file driver and
library version bounds setting. Should generally be the primary way for tests to obtain a File Access
Property List to use when testing with different VFDs would not be problematic.

\subsubsection{\texttt{h5\_fileaccess\_flags()}}

\begin{minted}{C}
hid_t
h5_fileaccess_flags(unsigned flags);
\end{minted}

Counterpart to \texttt{h5\_fileaccess()} which allows the caller to specify which parts of the File
Access Property List should be modified after it is created. \texttt{flags} may be specified as one of
the macro values \texttt{H5\_FILEACCESS\_VFD} or \texttt{H5\_FILEACCESS\_LIBVER}, where the former
specifies that the file driver of the FAPL should be modified, while the latter specifies that the
library version bounds setting of the FAPL should be modified. \texttt{flags} can also be specified
as a bit-wise OR of the two values, in which case behavior is equivalent to \texttt{h5\_fileaccess()}.

\subsubsection{\texttt{h5\_get\_vfd\_fapl()}}

\begin{minted}{C}
herr_t
h5_get_vfd_fapl(hid_t fapl_id);
\end{minted}

Modifies the File Access Property List specified in \texttt{fapl\_id} by setting a new Virtual File Driver
on it with default configuration values. The Virtual File Driver to be used is chosen according to the
value set for either the \texttt{HDF5\_DRIVER} or \texttt{HDF5\_TEST\_DRIVER} environment variable, with
preference given to \texttt{HDF5\_DRIVER}. These environment variables may be set to one of the following
strings:

\begin{itemize}

    \item "sec2" - Default sec2 driver; no configuration supplied
    \item "stdio" - C STDIO driver; no configuration supplied
    \item "core" - In-memory driver; 1MiB increment size and backing store enabled
    \item "core\_paged" - In-memory driver; 1MiB increment size and backing store enabled; write tracking
          enabled and 4KiB page size
    \item "split" - Multi driver with the file's metadata being placed into a file with a ".meta" suffix
          and the file's raw data being placed into a file with a ".raw" suffix
    \item "multi" - Multi driver with the file data being placed into several different files with the 
          suffixes "m.h5" (metadata), "s.h5" (superblock, userblock and driver info data), "b.h5" (b-tree data), "r.h5" (dataset raw data), "g.h5" (global heap data), "l.h5" (local heap data) and "o.h5"
          (object header data)
    \item "family" - Family driver with a family file size of 100MiB and with a default File Access Property
          List used for accessing the family file members. A different family file size can be specified
          in the environment variable as an integer value of bytes separated from the string "family" with whitespace, e.g. "family 52428800" would specify a family file size of 50MiB.
    \item "log" - Log driver using \texttt{stderr} for output, with the \texttt{H5FD\_LOG\_LOC\_IO} and
          \texttt{H5FD\_LOG\_ALLOC} flags set and 0 for the buffer size. A different set of flags may be
          specified for the driver as an integer value (corresponding to a bit-wise OR of flags) separated from the string "log" with whitespace, e.g. "log 14" would equate to the flag
          \texttt{H5FD\_LOG\_LOC\_IO}.
    \item "direct" - Direct driver with a 4KiB block size, a 32KiB copy buffer size and a 1KiB memory
          alignment setting. If the direct driver is not enabled when HDF5 is built,
          \texttt{h5\_get\_vfd\_fapl()} will return an error.
    \item "splitter" - Splitter driver using the default (sec2) VFD for both the read/write and write-only
          channels, an empty log file path and set to not ignore errors on the write-only channel
    \item "onion" - support not currently implemented; will cause \texttt{h5\_get\_vfd\_fapl()} to return an
          error.
    \item "subfiling" - Subfiling VFD with a default configuration of 1 I/O concentrator per node, a 32MiB
          stripe size and using the IOC VFD with 4 worker threads. \texttt{MPI\_COMM\_WORLD} and
          \texttt{MPI\_INFO\_NULL} are used for the MPI parameters.
    \item "mpio" - MPI I/O VFD with \texttt{MPI\_COMM\_WORLD} and \texttt{MPI\_INFO\_NULL} used for the MPI
          parameters.
    \item "mirror" - support not currently implemented; will cause \texttt{h5\_get\_vfd\_fapl()} to return an
          error.
    \item "hdfs" - support not currently implemented; will cause \texttt{h5\_get\_vfd\_fapl()} to return an
          error.
    \item "ros3" - support not currently implemented; will cause \texttt{h5\_get\_vfd\_fapl()} to return an
          error.

\end{itemize}

Other values for the environment variables will cause \texttt{h5\_get\_vfd\_fapl()} to return an error.

Returns a non-negative value on success and a negative value on failure.

\subsubsection{\texttt{h5\_get\_libver\_fapl()}}

\begin{minted}{C}
herr_t
h5_get_libver_fapl(hid_t fapl_id);
\end{minted}

Modifies the File Access Property List specified in \texttt{fapl\_id} by setting library version bound
values on it according to the value set for the \texttt{HDF5\_LIBVER\_BOUNDS} environment variable.
Currently, the only valid value for this environment variable is the string "latest", which will cause
this function to set the low and high version bounds both to \texttt{H5F\_LIBVER\_LATEST}. Other values
for the environment variable will cause this function to fail and return a negative value.

\subsubsection{\texttt{h5\_cleanup()}}
\label{apdx:h5test_h5cleanup}

\begin{minted}{C}
int
h5_cleanup(const char *base_name[], hid_t fapl);
\end{minted}

Used to cleanup temporary files created by a test program. \texttt{base\_name} is an array of filenames
without suffixes to clean up. The last entry in the array must be \texttt{NULL}. For each filename
specified, \texttt{h5\_cleanup()} will generate a VFD-dependent filename with \texttt{h5\_fixname()}
according to the given File Access Property List \texttt{fapl}, then call \texttt{H5Fdelete()} on the
resulting filename. \texttt{fapl} will be closed after all files are deleted. If the environment variable
\texttt{HDF5\_NOCLEANUP} has been defined, this function will have no effect and \texttt{fapl} will be
left open. Returns 1 if cleanup was performed and 0 otherwise.

\subsubsection{\texttt{h5\_delete\_all\_test\_files()}}

\begin{minted}{C}
void
h5_delete_all_test_files(const char *base_name[], hid_t fapl);
\end{minted}

Used to cleanup temporary files created by a test program. \texttt{base\_name} is an array of filenames
without suffixes to clean up. The last entry in the array must be \texttt{NULL}. For each filename
specified, this function will generate a VFD-dependent filename with \texttt{h5\_fixname()} according to
the given File Access Property List \texttt{fapl}, then call \texttt{H5Fdelete()} on the resulting filename.
\texttt{fapl} will \textbf{not} be closed after all files are deleted. \texttt{h5\_delete\_all\_test\_files()}
always performs file cleanup, regardless of if the \texttt{HDF5\_NOCLEANUP} environment variable has been defined.

\subsubsection{\texttt{h5\_delete\_test\_file()}}

\begin{minted}{C}
void
h5_delete_test_file(const char *base_name, hid_t fapl);
\end{minted}

Used to cleanup a temporary file created by a test program. \texttt{base\_name} is a filename without a
suffix to clean up. This function will generate a VFD-dependent filename with \texttt{h5\_fixname()}
according to the given File Access Property List \texttt{fapl}, then call \texttt{H5Fdelete()} on the resulting
filename. \texttt{fapl} will \textbf{not} be closed after the file is deleted.
\texttt{h5\_delete\_test\_file()} always performs file cleanup, regardless of if the
\texttt{HDF5\_NOCLEANUP} environment variable has been defined.

\subsubsection{\texttt{h5\_fixname()}}
\label{apdx:h5test_h5fixname}

\begin{minted}{C}
char
*h5_fixname(const char *base_name, hid_t fapl, char *fullname,
            size_t size);
\end{minted}

Given a base filename without a suffix, \texttt{base\_name}, and a File Access Property List, \texttt{fapl},
generates a filename according to the configuration set on \texttt{fapl}. The resulting filename is copied
to \texttt{fullname}, which is \texttt{size} bytes in size, including space for the \texttt{NUL} terminator.

\texttt{h5\_fixname()} is the primary way that tests should create filenames, as it accounts for the
possibility of a test being run with a non-default Virtual File Driver that may require a specialized
filename (e.g., the family driver). It also allows tests to easily output test files to a different
directory by setting the \texttt{HDF5\_PREFIX} (for serial tests) or \texttt{HDF5\_PARAPREFIX} (for parallel
tests) environment variable.

Returns the \texttt{fullname} parameter on success, or \texttt{NULL} on failure.

\subsubsection{\texttt{h5\_fixname\_superblock()}}

\begin{minted}{C}
char
*h5_fixname_superblock(const char *base_name, hid_t fapl, char *fullname,
                       size_t size);
\end{minted}

\texttt{h5\_fixname\_superblock()} is similar to \texttt{h5\_fixname()}, but generates the filename that
would need to be opened to find the logical HDF5 file's superblock. Useful for when a file is to be opened
with \texttt{open(2)} but the \texttt{h5\_fixname()} string contains stuff like format strings.

\subsubsection{\texttt{h5\_fixname\_no\_suffix()}}

\begin{minted}{C}
char
*h5_fixname_no_suffix(const char *base_name, hid_t fapl, char *fullname,
                      size_t size);
\end{minted}

\texttt{h5\_fixname\_no\_suffix()} is similar to \texttt{h5\_fixname()}, but generates a filename that
has no suffix, where the filename from \texttt{h5\_fixname()} would typically have ".h5".

\subsubsection{\texttt{h5\_fixname\_printf()}}

\begin{minted}{C}
char
*h5_fixname_printf(const char *base_name, hid_t fapl, char *fullname,
                   size_t size);
\end{minted}

\texttt{h5\_fixname\_printf()} is similar to \texttt{h5\_fixname()}, but generates a filename that can
be passed through a \texttt{printf}-style function to obtain the final, processed filename. Essentially
replaces all \% characters that would be used by a file driver with \%\%.

\subsubsection{\texttt{h5\_no\_hwconv()}}

\begin{minted}{C}
void
h5_no_hwconv(void);
\end{minted}

Temporarily turns off hardware datatype conversions in HDF5 during testing by calling
\texttt{H5Tunregister()} to unregister all the hard conversion pathways. Useful to verify that
datatype conversions for different datatypes still work correctly when emulated by the library.

\subsubsection{\texttt{h5\_rmprefix()}}

\begin{minted}{C}
const char
*h5_rmprefix(const char *filename);
\end{minted}

"Removes" a prefix from a filename by searching for the first occurence of ":" and returning
a pointer into the filename just past that occurrence. No actual changes are made to the file
name.

\subsubsection{\texttt{h5\_show\_hostname()}}

\begin{minted}{C}
void
h5_show_hostname(void);
\end{minted}

Prints out \texttt{hostname}-like information. Also prints out each MPI process' rank value if
HDF5 is built with parallel functionality enabled and MPI is initialized. Otherwise, if HDF5
is built with thread-safe functionality enabled and MPI is not initialized or HDF5 is not built
with parallel functionality enabled, also prints out thread ID values.

\subsubsection{\texttt{h5\_get\_file\_size()}}

\begin{minted}{C}
h5_stat_size_t
h5_get_file_size(const char *filename, hid_t fapl);
\end{minted}

Returns the size in bytes of the file with the given filename \texttt{filename}. A File Access Property
List specified for \texttt{fapl} will modify how the file size is calculated. If \texttt{H5P\_DEFAULT}
is passed for \texttt{fapl}, \texttt{stat} or the platform equivalent is used to determine the file size.
Otherwise, the calculation depends on the file driver set on \texttt{fapl}. For example, a FAPL setup
with the MPI I/O driver will cause \texttt{h5\_get\_file\_size()} to use \texttt{MPI\_File\_get\_size()},
while a FAPL setup with the family driver will cause \texttt{h5\_get\_file\_size()} to sum the sizes of
the files in the family file.

\subsubsection{\texttt{h5\_make\_local\_copy()}}

\begin{minted}{C}
int
h5_make_local_copy(const char *origfilename, const char *local_copy_name);
\end{minted}

Given a file with the filename \texttt{origfilename}, makes a byte-for-byte copy of the file, which is
then named \texttt{local\_copy\_name}, using POSIX I/O. Returns 0 on success and a negative value on
failure. This function is useful for making copies of test files that are under version control. Tests
should make a copy of the original file and then operate on the copy.

\subsubsection{\texttt{h5\_duplicate\_file\_by\_bytes()}}

\begin{minted}{C}
int
h5_duplicate_file_by_bytes(const char *orig, const char *dest);
\end{minted}

Similar to \texttt{h5\_make\_local\_copy()}, but uses C stdio functions. Returns 0 on success and a
negative value on failure.

\subsubsection{\texttt{h5\_compare\_file\_bytes()}}

\begin{minted}{C}
int
h5_compare_file_bytes(char *fname1, char *fname2);
\end{minted}

Performs a byte-for-byte comparison of two files with the names \texttt{fname1} and \texttt{fname2}.
Returns 0 if the files are identical and -1 otherwise.

\subsubsection{\texttt{h5\_verify\_cached\_stabs()}}

\begin{minted}{C}
herr_t
h5_verify_cached_stabs(const char *base_name[], hid_t fapl);
\end{minted}

Verifies that all groups in a set of files have their symbol table information cached, if present
and if their parent group also uses a symbol table. \texttt{base\_name} is an array of filenames
without suffixes, where the last entry must be \texttt{NULL}. \texttt{fapl} is the File Access
Property List used to open each of the files in \texttt{base\_name}. Returns a non-negative value
on success and a negative value on failure.

\subsubsection{\texttt{h5\_get\_dummy\_vfd\_class()}}

\begin{minted}{C}
H5FD_class_t
*h5_get_dummy_vfd_class(void);
\end{minted}

Allocates and returns a pointer to a "dummy" Virtual File Driver class which is generally
non-functional. Must be freed by the caller with \texttt{free()} once it is no longer needed.

\subsubsection{\texttt{h5\_get\_dummy\_vol\_class()}}

\begin{minted}{C}
H5VL_class_t
*h5_get_dummy_vol_class(void);
\end{minted}

Allocates and returns a pointer to a "dummy" Virtual Object Layer connector class which is
generally non-functional. Must be freed by the caller with \texttt{free()} once it is no longer
needed.

\subsubsection{\texttt{h5\_get\_version\_string()}}

\begin{minted}{C}
const char
*h5_get_version_string(H5F_libver_t libver);
\end{minted}

Given a particular library version bound value, \texttt{libver}, translates the value into
a canonical string value that is returned.

\subsubsection{\texttt{h5\_check\_if\_file\_locking\_enabled()}}

\begin{minted}{C}
herr_t
h5_check_if_file_locking_enabled(bool *are_enabled);
\end{minted}

Checks if file locking is enabled on the system by creating a temporary file and calling
\texttt{flock()} or the platform equivalent on it. A non-negative value is return on success
and a negative value is returned otherwise. If this function succeeds and \texttt{are\_enabled}
is set to \texttt{true}, file locking is enabled on the system. Otherwise, it should be assumed
the file locking is not enabled or is problematic.

\subsubsection{\texttt{h5\_check\_file\_locking\_env\_var()}}

\begin{minted}{C}
void
h5_check_file_locking_env_var(htri_t *use_locks,
                              htri_t *ignore_disabled_locks);
\end{minted}

Parses the value of the \texttt{HDF5\_USE\_FILE\_LOCKING} environment variable, if set, and
returns whether or not file locking should be used and whether or not failures should be
ignored when attempting to use file locking on a system where it is disabled.

\subsubsection{\texttt{h5\_using\_native\_vol()}}
\label{apdx:h5test_h5usingnativevol}

\begin{minted}{C}
herr_t
h5_using_native_vol(hid_t fapl_id, hid_t obj_id, bool *is_native_vol);
\end{minted}

Checks if the VOL connector being used for testing is the library's native VOL connector. One of
either \texttt{fapl\_id} or \texttt{obj\_id} must be provided as a reference point to be checked;
if both are provided, checking of \texttt{obj\_id} takes precedence. \texttt{H5I\_INVALID\_HID}
should be specified for the parameter that is not provided.

\texttt{obj\_id} must be the ID of an HDF5 object that is accessed with the VOL connector to check.
If \texttt{obj\_id} is provided, the entire VOL connector stack is checked to see if it resolves
to the native VOL connector. If only \texttt{fapl\_id} is provided, only the top-most VOL connector
set on \texttt{fapl\_id} is checked against the native VOL connector.

A non-negative value is return on success and a negative value is returned otherwise.

\subsubsection{\texttt{h5\_get\_test\_driver\_name()}}

\begin{minted}{C}
const char
*h5_get_test_driver_name(void);
\end{minted}

Returns a pointer to the name of the VFD being used for testing. If the environment variable
\texttt{HDF5\_DRIVER} or \texttt{HDF5\_TEST\_DRIVER} has been set, the value set for that variable
is returned, with preference given to the \texttt{HDF5\_DRIVER} environment variable if both are set.
Otherwise, the name of the library's default VFD is returned.

\subsubsection{\texttt{h5\_using\_default\_driver()}}

\begin{minted}{C}
bool
h5_using_default_driver(const char *drv_name);
\end{minted}

Returns \texttt{true} if the name of the VFD being used for testing matches the name of the
library's default VFD and \texttt{false} otherwise. If \texttt{drv\_name} is \texttt{NULL},
\texttt{h5\_get\_test\_driver\_name()} is called to obtain the name of the VFD in use before
making the comparison.

\subsubsection{\texttt{h5\_using\_parallel\_driver()}}

\begin{minted}{C}
herr_t
h5_using_parallel_driver(hid_t fapl_id, bool *driver_is_parallel);
\end{minted}

Checks if the VFD set on \texttt{fapl\_id} is a parallel-enabled VFD that supports MPI. A VFD must
have set the \texttt{H5FD\_FEAT\_HAS\_MPI} feature flag to be considered as a parallel-enabled VFD.
\texttt{fapl\_id} may be \texttt{H5P\_DEFAULT}. A non-negative value is return on success and a negative
value is returned otherwise.

\subsubsection{\texttt{h5\_driver\_is\_default\_vfd\_compatible()}}

\begin{minted}{C}
herr_t
h5_driver_is_default_vfd_compatible(hid_t fapl_id,
                                    bool *default_vfd_compatible);
\end{minted}

Checks if the VFD set on \texttt{fapl\_id} creates files that are compatible with the library's
default VFD. For example, the core and MPI I/O drivers create files that are compatible with the
library's default VFD, while the multi and family drivers do not since they split the HDF5 file
into several different files. This check is helpful for skipping tests that use pre-generated testing
files. VFDs that create files which aren't compatible with the default VFD will generally not be able
to open these pre-generated files and those particular tests will fail.

\texttt{fapl\_id} may be \texttt{H5P\_DEFAULT}. A non-negative value is return on success and a negative
value is returned otherwise.

\subsubsection{\texttt{h5\_driver\_uses\_multiple\_files()}}

\begin{minted}{C}
bool
h5_driver_uses_multiple_files(const char *drv_name, unsigned flags);
\end{minted}

Returns \texttt{true} if the given VFD name, \texttt{drv\_name}, matches the name of a VFD which
stores data using multiple files, according to the specified \texttt{flags} and \texttt{false} otherwise.
If \texttt{drv\_name} is \texttt{NULL}, the \texttt{h5\_get\_test\_driver\_name()} function is called to
obtain the name of the VFD in use before making the comparison. The values for \texttt{flags} are as
follows:

\begin{itemize}

    \item \texttt{H5\_EXCLUDE\_MULTIPART\_DRIVERS} - This flag excludes any drivers which store data
          using multiple files which, together, make up a single logical file. These are drivers like
          the split, multi and family drivers.
    \item \texttt{H5\_EXCLUDE\_NON\_MULTIPART\_DRIVERS} - This flag excludes any drivers which store
          data using multiple files which are separate logical files. The splitter driver is an example
          of this type of driver.

\end{itemize}

\subsubsection{\texttt{h5\_local\_rand()}}

\begin{minted}{C}
int
h5_local_rand(void);
\end{minted}

Function to return a random number without modifying state for the \texttt{rand()}/\texttt{random()}
functions.

\subsubsection{\texttt{h5\_local\_srand()}}

\begin{minted}{C}
void
h5_local_srand(unsigned int seed);
\end{minted}

Function to seed the \texttt{h5\_local\_rand()} function without modifying state for the \texttt{rand()}/\texttt{random()} functions.

\subsubsection{\texttt{h5\_szip\_can\_encode()}}

\begin{minted}{C}
int
h5_szip_can_encode(void);
\end{minted}

Returns a value that indicates whether or not the library's SZIP filter has encoding/decoding enabled.
Returns 1 if encoding and decoding are enabled. Returns 0 if only decoding is enabled. Otherwise, returns
-1.

\subsubsection{\texttt{h5\_set\_info\_object()}}

\begin{minted}{C}
int
h5_set_info_object(void);
\end{minted}

Utility function for parallel HDF5 tests which parses the \texttt{HDF5\_MPI\_INFO} environment variable
for ";"-delimited key=value pairs and sets them on the \texttt{h5\_io\_info\_g} MPI Info global variable for later use by testing. Returns 0 on success and a negative value otherwise.

\subsubsection{\texttt{h5\_dump\_info\_object()}}

\begin{minted}{C}
void
h5_dump_info_object(MPI_Info info);
\end{minted}

Given an MPI Info object, \texttt{info}, iterates through all the keys set on the Info object and
prints them out as key=value pairs.

\subsubsection{\texttt{getenv\_all()}}

\begin{minted}{C}
char
*getenv_all(MPI_Comm comm, int root, const char *name);
\end{minted}

Retrieves the value of the environment variable \texttt{name}, if set, on the MPI process with rank
value \texttt{root} on the MPI Communicator \texttt{comm}, then broadcasts the result to other MPI
processes in \texttt{comm}. Collective across the MPI Communicator specified in \texttt{comm}.
If MPI is not initialized, simply calls \texttt{getenv(name)} and returns a pointer to the result.
\texttt{NULL} is returned if the environment variable \texttt{name} is not set.

\textbf{Note:} the pointer returned by this function is only valid until the next call to
\texttt{getenv\_all()} and the data stored there must be copied somewhere else before any further calls
to \texttt{getenv\_all()} take place.

\subsubsection{\texttt{h5\_send\_message()}}

\begin{minted}{C}
void
h5_send_message(const char *file, const char *arg1, const char *arg2);
\end{minted}

Utility function to facilitate inter-process communication by "sending" a message with a temporary
file. \texttt{file} is the name of the temporary file to be created. \texttt{arg1} and \texttt{arg2}
are strings to be written to the first and second lines of the file, respectively, and may both be
\texttt{NULL}.

\subsubsection{\texttt{h5\_wait\_message()}}

\begin{minted}{C}
herr_t
h5_wait_message(const char *file);
\end{minted}

Utility function to facilitate inter-process communication by waiting until a temporary file written by
the \texttt{h5\_send\_message()} function is available for reading. \texttt{file} is the name of the file
being waited on and should match the filename provided to \texttt{h5\_send\_message()}. This
function repeatedly attempts to open a file with the given filename until it is either successful
or times out. The temporary file is removed once it has been successfully opened. A non-negative value
is return on success and a negative value is returned otherwise.

\end{document}