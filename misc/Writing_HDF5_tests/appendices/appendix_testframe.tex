\documentclass[../HDF5_RFC.tex]{subfiles}
 
\begin{document}

% Start each appendix on a new page
\newpage

\section{Appendix: 'Testframe' testing framework}
\label{apdx:test_frame}

The following sections give an overview of HDF5's 'testframe' testing framework which is contained
within the \texttt{test/testframe.h} and \texttt{test/testframe.c} files. Details are only given
for the types, functions, etc. that appear in the relevant header files.

\subsection{Macros}

\begin{minted}{C}
#define MAXTESTNAME 64
\end{minted}

The maximum number of bytes in a test name string, including the NUL terminator.

\begin{minted}{C}
#define MAXTESTDESC 128
\end{minted}

The maximum number of bytes in a test description string, including the NUL terminator.

\begin{minted}{C}
/* Number of seconds to wait before killing a test (requires alarm(2)) */
#define H5_ALARM_SEC 1200 /* default is 20 minutes */
\end{minted}

The default compile-time value for the \texttt{TestAlarmOn()} function.

\begin{minted}{C}
/*
 * Predefined test verbosity levels.
 *
 * Convention:
 *
 * The higher the verbosity value, the more information printed.
 * So, output for higher verbosity also include output of all lower
 * verbosity.
 *
 *  Value     Description
 *  0         None:   No informational message.
 *  1                 "All tests passed"
 *  2                 Header of overall test
 *  3         Default: header and results of individual test
 *  4
 *  5         Low:    Major category of tests.
 *  6
 *  7         Medium: Minor category of tests such as functions called.
 *  8
 *  9         High:   Highest level.  All information.
 */
#define VERBO_NONE 0 /* None    */
#define VERBO_DEF  3 /* Default */
#define VERBO_LO   5 /* Low     */
#define VERBO_MED  7 /* Medium  */
#define VERBO_HI   9 /* High    */
\end{minted}

Macros for the different test verbosity levels, for use with the \texttt{SetTestVerbosity()} function.

\begin{minted}{C}
/*
 * Verbose queries
 * Only None needs an exact match.  The rest are at least as much.
 */
#define VERBOSE_NONE (HDGetTestVerbosity() == VERBO_NONE)
#define VERBOSE_DEF  (HDGetTestVerbosity() >= VERBO_DEF)
#define VERBOSE_LO   (HDGetTestVerbosity() >= VERBO_LO)
#define VERBOSE_MED  (HDGetTestVerbosity() >= VERBO_MED)
#define VERBOSE_HI   (HDGetTestVerbosity() >= VERBO_HI)
\end{minted}

Macros for use with the \texttt{GetTestVerbosity()} function that return whether or not the current
test verbosity level is at least set to the given level.

\begin{minted}{C}
#define SKIPTEST  1 /* Skip this test */
#define ONLYTEST  2 /* Do only this test */
#define BEGINTEST 3 /* Skip all tests before this test */
\end{minted}

Macros for values passed to the \texttt{SetTest()} function that determine the final behavior for a
test that has been added to the list of tests.

\begin{minted}{C}
#define MESSAGE(V, A)
do {
    if (TestFrameworkProcessID_g == 0 && TestVerbosity_g > (V))
        printf A;
} while(0)
\end{minted}

Macro to print a message string \texttt{A} from the process with the process ID of 0, as long as the current test verbosity level setting is greater than \texttt{V}.

\subsection{Enumerations}

None defined

\subsection{Structures}

None defined

\subsection{Global variables}

\begin{minted}{C}
int TestFrameworkProcessID_g = 0;
\end{minted}

Contains an ID value for the current process which can be set during calls to \texttt{TestInit()}.
Primarily used to control output from a parallel test so that it only comes from one MPI process,
but also used for control flow.

\begin{minted}{C}
int TestVerbosity_g = VERBO_DEF;
\end{minted}

Contains the current setting for the level of output that tests should print. Defined values range
from 0 - 9, with higher values implying more verbose output.

\subsection{Functions}

\subsubsection{\texttt{TestInit()}}
\label{apdx:testframe_testinit}

\begin{minted}{C}
herr_t
TestInit(const char *ProgName, void (*TestPrivateUsage)(FILE *stream),
         int (*TestPrivateParser)(int argc, char *argv[]),
         herr_t (*TestSetupFunc)(void), herr_t (*TestCleanupFunc)(void),
         int TestProcessID);
\end{minted}

Initializes test program-specific information and infrastructure. \texttt{TestInit()} should be
called before any other function from this testing framework is called, but after other optional
library setup functions, such as \texttt{H5open()} or \texttt{H5dont\_atexit()}.

\texttt{ProgName} is simply a name for the test program (which may be different than the actual
executable's name).

\texttt{TestPrivateUsage} is a pointer to a function that can print out additional test program usage
help text that is specific to the test program. That function will be called by the \texttt{TestUsage()}
function after it has printed out the more general test program command-line option information. May
be \texttt{NULL}.

\texttt{TestPrivateParser} is a pointer to a function that will parse any command-line options that are
specific to a test program. That function will be called by the \texttt{TestParseCmdLine()} function when
a non-standard command-line option is found. May be \texttt{NULL}.

\texttt{TestSetupFunc} is a pointer to a function that can be used to setup any state needed before tests
begin executing. If provided, this callback function will be called as part of \texttt{TestInit()} once
the testing framework has been fully initialized. May be \texttt{NULL}.

\texttt{TestCleanupFunc} is a pointer to a function that can be used to clean up any state after tests
have finished executing. If provided, this callback function will be called by \texttt{TestShutdown()}
before the testing framework starts being shut down. May be \texttt{NULL}.

\subsubsection{\texttt{TestShutdown()}}

\begin{minted}{C}
herr_t
TestShutdown(void);
\end{minted}

Performs tear-down of the testing infrastructure by cleaning up internal state needed for running
tests and freeing any associated memory. \texttt{TestShutdown()} should be called after any other function
from this testing framework is called, and just before any optional library shutdown functions, such as
\texttt{H5close()}.

\subsubsection{\texttt{TestUsage()}}

\begin{minted}{C}
void
TestUsage(FILE *stream);
\end{minted}

Prints out the usage help text for the test program to the given \texttt{FILE} pointer. Prints out a
list of the general command-line options for the program, an optional list of additional test
program-specific command-line options (if a \texttt{TestPrivateUsage} callback function was specified)
and a list of all the tests and test descriptions for the test program. \texttt{stream} may be \texttt{NULL},
in which case \texttt{stdout} is used.

\textbf{Note:} when a parallel test calls \texttt{TestUsage()}, the output, including additional output
from the optional callback specified in \texttt{TestInit()}, is only printed from the MPI process with rank
value 0. Any collective operations should currently be avoided in the optional callback if one is provided.

\subsubsection{\texttt{TestInfo()}}

\begin{minted}{C}
void
TestInfo(FILE *stream);
\end{minted}

Prints out miscellaneous test information, which currently only includes the version of the HDF5
library that the test program was linked to, to the given \texttt{FILE} pointer. \texttt{stream} may be \texttt{NULL}, in which case \texttt{stdout} is used.

\textbf{Note:} when a parallel test calls \texttt{TestInfo()}, the output is only printed from the MPI
process with rank value 0.

\subsubsection{\texttt{AddTest()}}

\begin{minted}{C}
herr_t
AddTest(const char *TestName, void (*TestFunc)(const void *),
        void (*TestSetupFunc)(void *), void (*TestCleanupFunc)(void *),
        const void *TestData, size_t TestDataSize, const char *TestDescr);
\end{minted}

This function adds a test to the list of tests that will be run when the \texttt{PerformTests()}
function is called.

\texttt{TestName} is a short name string for the test and must be \texttt{MAXTESTNAME} bytes or less,
including the NUL terminator. If the specified string begins with the character '-', the test will
be set to be skipped by default.

\texttt{TestFunc} is a pointer to the function that will be called for the test. The function must
return no value and accept a single \texttt{const void *} as an argument, which will point to any
parameters to be passed to the test that are specified in \texttt{TestData}.

\texttt{TestSetupFunc} is an optional pointer to a function that will be called before the test's
main test function is called. The function must return no value and accept a single \texttt{void *}
as an argument, which will point to any parameters to be passed to the test that are specified in
\texttt{TestData}.

\texttt{TestCleanupFunc} is an optional pointer to a function that will be called after the test's
main test function has finished executing. The function must return no value and accept a single
\texttt{void *} as an argument, which will point to any parameters to be passed to the test that are
specified in \texttt{TestData}.

\texttt{TestData} is an optional pointer to test parameters that will be passed to the test's main
test function when executed, as well as the test's optional setup and cleanup callbacks. If given,
the testing framework will make a copy of the parameters according to the size specified in
\texttt{TestDataSize}. If \texttt{TestData} is not \texttt{NULL}, \texttt{TestDataSize} must be a
positive value. Otherwise, if \texttt{TestData} is \texttt{NULL}, \texttt{TestDataSize} must be 0.

\texttt{TestDataSize} is the size of the test parameter data to be passed to the test's main test
function and setup and callback functions during execution. If \texttt{TestData} is not \texttt{NULL},
\texttt{TestDataSize} must be a positive value. Otherwise, if \texttt{TestData} is \texttt{NULL},
\texttt{TestDataSize} must be 0.

\texttt{TestDescr} is a short description string for the test and must be \texttt{MAXTESTDESC} bytes
or less, including the NUL terminator. The string passed in \texttt{TestDescr} may be an empty string,
but it is advised that test authors give a description to a test.

\subsubsection{\texttt{TestParseCmdLine()}}

\begin{minted}{C}
herr_t
TestParseCmdLine(int argc, char *argv[]);
\end{minted}

Parses the command-line options for the test program. Parse standard command-line options until it
encounters a non-standard option, at which point it delegates to the \texttt{TestPrivateParser} callback
function if one was specified by the \texttt{TestInit()} function. The following command-line options
are the current standard options:

\begin{itemize}

    \item \texttt{-verbose} / \texttt{-v} - Used to specify the level of verbosity of output from the
          test program. An extra argument must be provided to set the level, with the following being
          acceptable values: 'l' - low verbosity (value 5), 'm' - medium verbosity (value 7),
          'h' - high verbosity (value 9), any number in 'int' range - set the verbosity to that value
    \item \texttt{-exclude} / \texttt{-x} - Used to specify a test that should be excluded when running.
          The short name of the test (as provided to \texttt{AddTest()}) must be provided as an extra
          argument to the command-line option.
    \item \texttt{-begin} / \texttt{-b} - Used to specify that a particular test should be the first
          test run from the set of tests. All tests before that test (in the order added by
          \texttt{AddTest()}) will be skipped and all tests after that test will be run as normal.
          The short name of the test (as provided to \texttt{AddTest()}) must be provided as an extra
          argument to the command-line option.
    \item \texttt{-only} / \texttt{-o} - Used to specify that only a particular test should be run.
          All other tests will be skipped. The short name of the test (as provided to \texttt{AddTest()})
          must be provided as an extra argument to the command-line option.
    \item \texttt{-summary} / \texttt{-s} - Used to specify that a summary of all tests run should be
          printed out before the test program exits. No extra arguments should be provided.
    \item \texttt{-disable-error-stack} - Used to specify that printing out of HDF5 error stacks should
          be disabled. No extra arguments should be provided.
    \item \texttt{-help} / \texttt{-h} - Used to print out the usage help text of the test program.
    \item \texttt{-cleanoff} / \texttt{-c} - Used to specify that the \texttt{TestCleanupFunc()} callback
          function for each test should not clean up temporary HDF5 files when the test program exits.
          No extra arguments should be provided.
    \item \texttt{-maxthreads} / \texttt{-t} - Used to specify a maximum number of threads that a test
          program should be allowed to spawn in addition to the main thread. If a negative value is
          specified, test programs will be allowed to spawn any number of threads. The value 0 indicates
          that no additional threads should be spawned.

\end{itemize}

\textbf{Note:} \texttt{TestParseCmdLine()} requires that all standard command-line arguments must appear
before any non-standard arguments that would be parsed by an optional argument parsing callback function specified in \texttt{TestInit()}.

\textbf{Note:} \texttt{TestParseCmdLine()} should not be called until all tests have been added by
\texttt{AddTest()} since some of the command-line arguments that are parsed involve the ability to skip
certain tests.

\subsubsection{\texttt{PerformTests()}}

\begin{minted}{C}
herr_t
PerformTests(void);
\end{minted}

Runs all tests added to the list of tests with \texttt{AddTest()} that aren't set to be skipped. For
each test, the test's setup callback function (if supplied) will be called first, followed by the test's
primary function and then the test's cleanup callback function (if supplied). Before each test begins,
a timer is enabled by a call to \texttt{TestAlarmOn()} to prevent the test from running longer than
desired. A call to \texttt{TestAlarmOff()} disables this timer after each test has finished.

\subsubsection{\texttt{TestSummary()}}

\begin{minted}{C}
void
TestSummary(FILE *stream);
\end{minted}

Used to print out a short summary after tests have run, which includes the name and description of
each test and the number of errors that occurred while running that test. If a test was skipped, the
number of errors for that test will show as "N/A". \texttt{stream} may be \texttt{NULL}, in which case
\texttt{stdout} is used.

\textbf{Note:} when a parallel test calls \texttt{TestSummary()}, the output is only printed from the
MPI process with rank value 0.

\subsubsection{\texttt{GetTestVerbosity()}}

\begin{minted}{C}
int
GetTestVerbosity(void);
\end{minted}

Returns the current test verbosity level setting.

\subsubsection{\texttt{SetTestVerbosity()}}

\begin{minted}{C}
int
SetTestVerbosity(int newval);
\end{minted}

Sets the current test verbosity level to the value specified by \texttt{newval}. If \texttt{newval} is negative, the test verbosity level is set to the lowest value (\texttt{VERBO\_NONE}). If \texttt{newval}
is greater than the highest verbosity value, it is set to the highest verbosity value (\texttt{VERBO\_HI}). The function returns the previous test verbosity level setting.

\subsubsection{\texttt{ParseTestVerbosity()}}

\begin{minted}{C}
herr_t
ParseTestVerbosity(char *argv);
\end{minted}

Parses a string for a test verbosity level value, then sets the test verbosity level to that value.
The string may be the character 'l' (for low verbosity), 'm' (for medium verbosity), 'h' (for high verbosity)
or a number between 0-9, corresponding to the different predefined levels of test verbosity. If a negative
number is specified, the test verbosity level is set to the default (\texttt{VERBO\_DEF}). If a number
greater than \texttt{VERBO\_HI} is specified, the test verbosity level is set to \texttt{VERBO\_HI}. If \texttt{ParseTestVerbosity()} can't parse the string, a negative value will be returned to indicate
failure.

\subsubsection{\texttt{GetTestExpress()}}
\label{apdx:testframe_gettestexpress}

\begin{minted}{C}
int
GetTestExpress(void);
\end{minted}

Returns the current \texttt{TestExpress} level setting from the 'h5test' framework. If the
\texttt{TestExpress} level has not yet been set, it will be initialized to the default value
(currently level 1, unless overridden at configuration time).

\subsubsection{\texttt{SetTestExpress()}}

\begin{minted}{C}
void
SetTestExpress(int newval);
\end{minted}

Sets the current \texttt{TestExpress} level setting to the value specified by \texttt{newval}. If \texttt{newval} is negative, the \texttt{TestExpress} level is set to the default value (currently
level 1, unless overridden at configuration time). If \texttt{newval} is greater than the highest
\texttt{TestExpress} level (3), it is set to the highest \texttt{TestExpress} level.

\subsubsection{\texttt{GetTestSummary()}}

\begin{minted}{C}
bool
GetTestSummary(void);
\end{minted}

Returns whether or not a test program should call \texttt{TestSummary()} to print out a summary of
test results after tests have run.

\subsubsection{\texttt{GetTestCleanup()}}
\label{apdx:testframe_gettestcleanup}

\begin{minted}{C}
int
GetTestCleanup(void);
\end{minted}

Returns whether or not a test should clean up any temporary files it has created when it is finished
running. If \texttt{true} is returned, the test should clean up temporary files. Otherwise, it should
leave them in place. Each test that has a cleanup callback should call \texttt{GetTestCleanup()} in
that callback to determine whether or not to clean up temporary files.

\subsubsection{\texttt{SetTestNoCleanup()}}

\begin{minted}{C}
void
SetTestNoCleanup(void);
\end{minted}

Sets the temporary test file cleanup status to "don't cleanup temporary files", causing future calls to \texttt{GetTestCleanup()} to return \texttt{false} and inform tests that they should not clean up temporary test files they have created.

\subsubsection{\texttt{GetTestNumErrs()}}

\begin{minted}{C}
int
GetTestNumErrs(void);
\end{minted}

Returns the total number of errors recorded during the execution of the test program. This number is
primarily used to determine whether the test program should exit with a success or failure value.

\subsubsection{\texttt{IncTestNumErrs()}}

\begin{minted}{C}
void
IncTestNumErrs(void);
\end{minted}

Simply increments the number of errors recorded for the test program.

\subsubsection{\texttt{TestErrPrintf()}}

\begin{minted}{C}
int
TestErrPrintf(const char *format, ...);
\end{minted}

Wrapper around \texttt{vfprintf} that includes a call to \texttt{IncTestNumErrs()}. The function
returns the return value of \texttt{vfprintf}.

\subsubsection{\texttt{SetTest()}}

\begin{minted}{C}
herr_t
SetTest(const char *testname, int action);
\end{minted}

Given a test's short name and an action, modifies the behavior of how a test will run. The acceptable
values for \texttt{action} are:

\begin{itemize}

    \item \texttt{SKIPTEST} (1) - Skip this test.
    \item \texttt{ONLYTEST} (2) - Set this test to be the only test that runs. All other tests will
          be set to be skipped.
    \item \texttt{BEGINTEST} (3) - Set this test so that it will be the first test in the ordering
          that will run. All tests before it in the ordering will be set to be skipped and all tests
          after it will be untouched.

\end{itemize}

Other values for \texttt{action} will cause \texttt{SetTest()} to return a negative value for failure.

Multiple tests can be set to the value \texttt{ONLYTEST} in order to run a subset of tests. This is
intended as a convenient alternative to needing to skip many other tests by setting them to the value
\texttt{SKIPTEST}.

\subsubsection{\texttt{GetTestMaxNumThreads()}}
\label{apdx:testframe_gettestmaxnumthreads}

\begin{minted}{C}
int
GetTestMaxNumThreads(void);
\end{minted}

Returns the value for the maximum number of threads that a test program is allowed to spawn in addition
to the main thread. This number is usually configured by a command-line argument passed to
the test program and is intended for allowing tests to adjust their workload according to the resources
of the testing environment.

The default value is -1, meaning that multi-threaded tests aren't limited in the number of threads they
can spawn, but should still only use a reasonable amount of threads. The value 0 indicates that no additional threads should be spawned, which is primarily for testing purposes. The value returned by \texttt{GetTestMaxNumThreads()} is meaningless for non-multi-threaded tests.

\subsubsection{\texttt{SetTestMaxNumThreads()}}

\begin{minted}{C}
herr_t
SetTestMaxNumThreads(int max_num_threads);
\end{minted}

Sets the value for the maximum number of threads a test program is allowed to spawn in addition to the
main thread for the test program. This number is usually configured by a command-line argument passed to
the test program and is intended for allowing tests to adjust their workload according to the resources
of the testing environment.

If \texttt{max\_num\_threads} is a negative value, test programs will be allowed to spawn any number of
threads, though it is advised that test programs try to limit this to a reasonable number. The value 0
indicates that no additional threads should be spawned, which is primarily for testing purposes.

\subsubsection{\texttt{TestAlarmOn()}}
\label{apdx:testframe_testalarmon}

\begin{minted}{C}
herr_t
TestAlarmOn(void);
\end{minted}

Enables a timer for the test program using \texttt{alarm(2)}. If \texttt{alarm(2)} support is not available
(the macro \texttt{H5\_HAVE\_ALARM} is not defined), the function does nothing. The default value passed to \texttt{alarm(2)} is defined by the macro \texttt{H5\_ALARM\_SEC} (currently 1200 seconds). If the
\texttt{HDF5\_ALARM\_SECONDS} environment variable is defined, its value will be parsed by
\texttt{strtoul()} and will override the default value.

\subsubsection{\texttt{TestAlarmOff()}}
\label{apdx:testframe_testalarmoff}

\begin{minted}{C}
void
TestAlarmOff(void);
\end{minted}

Disables any previously set timer for a test program by calling \texttt{alarm(2)} with a value of 0. If
\texttt{alarm(2)} support is not available (the macro \texttt{H5\_HAVE\_ALARM} is not defined), the
function does nothing.

\end{document}