\documentclass[../HDF5_RFC.tex]{subfiles}
 
\begin{document}

% Start each appendix on a new page
\newpage

\section{Appendix: Examples}
\label{apdx:examples}

Error checking is omitted in the following examples for simplicity.

\subsection{\textbf{Example skeleton for a 'testframe' HDF5 test}}
\label{apdx:testframe_example}

\begin{minted}{C}
#include "testframe.h"

static herr_t
test1(TestParams_t *params)
{
    int *test_data = (int *)params->UserParams;

    if (unsupported_functionality)
        return SKIP;

    /* Do test */
    ...

    return SUCCEED;

error:
    return FAIL;
}

static herr_t
test2(TestParams_t *params)
{
    int *test_data = (int *)params->UserParams;

    if (unsupported_functionality)
        return SKIP;

    /* Do test */
    ...

    return SUCCEED;

error:
    return FAIL;
}

int
main(int argc, char **argv)
{
    int test1_data = 1;
    int test2_data = 2;

    H5open();

    /* Initialize testing framework */
    TestInit(argv[0], NULL, NULL, NULL, NULL, 0, 0);

    /* Display testing information */
    TestInfo(stdout);

    /* Add each test */
    AddTest("test1", test1, test1_setup, test1_cleanup,
            &test1_data, sizeof(test1_data), 0, "Runs test 1");
    AddTest("test2", test2, test2_setup, test2_cleanup,
            &test2_data, sizeof(test2_data), 0, "Runs test 2");

    /* Parse command line arguments */
    TestParseCmdLine(argc, argv);

    /* Run tests */
    PerformTests();

    /* Display test summary, if requested */
    if (GetTestSummary())
        TestSummary(stdout);

    /* If test file cleanup is not disabled, cleanup any
       test files */
    if (GetTestCleanup())
        ...;

    /* Release test infrastructure */
    TestShutdown();

    /* Shut down HDF5 */
    H5close();

    if (GetTestNumErrs() > 0 || GetTestsFailedCount() > 0)
        exit(EXIT_FAILURE);
    else
        exit(EXIT_SUCCESS);
}
\end{minted}

\subsection{\textbf{Example skeleton for a parallel 'testframe' HDF5 test}}
\label{apdx:testframe_parallel_example}

\begin{minted}{C}
#include "testframe.h"

static herr_t
test1(TestParams_t *params)
{
    int *test_data = (int *)params->UserParams;

    if (unsupported_functionality)
        return SKIP;

    /* Do test */
    ...

    return SUCCEED;

error:
    return FAIL;
}

static herr_t
test2(TestParams_t *params)
{
    int *test_data = (int *)params->UserParams;

    if (unsupported_functionality)
        return SKIP;

    /* Do test */
    ...

    return SUCCEED;

error:
    return FAIL;
}

int
main(int argc, char **argv)
{
    int test1_data = 1;
    int test2_data = 2;
    int num_errs   = 0;
    int mpi_size; /* MPI communicator size */
    int mpi_rank; /* MPI process rank value */

    MPI_Init(&argc, &argv);

    MPI_Comm_rank(MPI_COMM_WORLD, &mpi_rank);
    MPI_Comm_size(MPI_COMM_WORLD, &mpi_size);

    H5open();

    /* Initialize testing framework; pass MPI rank value
       for process ID */
    TestInit(argv[0], NULL, NULL, NULL, NULL, 0, mpi_rank);

    /* Display testing information */
    TestInfo(stdout);

    /* Add each test */
    AddTest("test1", test1, test1_setup, test1_cleanup,
            &test1_data, sizeof(test1_data), 0, "Runs test 1");
    AddTest("test2", test2, test2_setup, test2_cleanup,
            &test2_data, sizeof(test2_data), 0, "Runs test 2");

    /* Parse command line arguments */
    TestParseCmdLine(argc, argv);

    /* Run tests */
    PerformTests();

    /* Display test summary, if requested */
    if (GetTestSummary())
        TestSummary(stdout);

    /* If test file cleanup is not disabled, cleanup any
       test files */
    if (GetTestCleanup())
        ...;

    /* Gather errors from all processes */
    /* NOTE: this isn't strictly necessary as MPI will
       handle any process exiting with a non-zero status,
       but it is good practice. It may also be useful to
       gather the count of failed tests from all MPI
       processes with GetTestsFailedCount() and check
       that as well, as in the serial test program
       example. */
    num_errs = GetTestNumErrs();
    MPI_Allreduce(MPI_IN_PLACE, &num_errs, 1, MPI_INT,
                  MPI_MAX, MPI_COMM_WORLD);

    /* Release test infrastructure */
    TestShutdown();

    /* Shut down HDF5 */
    H5close();

    MPI_Finalize();

    if (num_errs > 0)
        exit(EXIT_FAILURE);
    else
        exit(EXIT_SUCCESS);
}
\end{minted}

\subsection{\textbf{Example skeleton for a multi-threaded HDF5 test}}
\label{apdx:multithread_example}

Note that the details of this skeleton program are subject to change as the 'testframe'
testing framework is adapted to work with multi-threaded tests.

\begin{minted}{C}
#include "testframe.h"

/* Allow 10 seconds of margin for timed-out tests to cleanup */
#define MARGIN 10

static herr_t
test1(TestParams_t *params)
{
    unsigned max_num_threads = GetTestMaxNumThreads();
    unsigned timeout         = *(unsigned *)params->UserParams;

    if (unsupported_functionality)
        return SKIP;

    /* Calculate amount of work to do based on timeout */
    ...

    /* Do test */
    ...

    return SUCCEED;

error:
    return FAIL;
}

static herr_t
test2(TestParams_t *params)
{
    unsigned max_num_threads = GetTestMaxNumThreads();
    unsigned timeout         = *(unsigned *)params->UserParams;

    if (unsupported_functionality)
        return SKIP;

    /* Calculate amount of work to do based on timeout */
    ...

    /* Do test */
    ...

    return SUCCEED;

error:
    return FAIL;
}

int
main(int argc, char **argv)
{
    unsigned runtime;
    unsigned num_tests;
    unsigned test_timeout;
    int      TestExpress;

    H5open();

    /* Initialize testing framework, passing H5_MULTITHREAD_TEST
       flag */
    TestInit(argv[0], NULL, NULL, NULL, NULL, H5_MULTITHREAD_TEST, 0);

    /* Display testing information */
    TestInfo(stdout);

    /* Calculate number of tests somehow */
    num_tests = 2;

    /* Adjust test program runtime based on TestExpress level */
    TestExpress = GetTestExpress();
    if (TestExpress == 0) {
        runtime = 0; /* Run with no timeout (runtime = 0) or pick a
                        reasonable timeout */
    }
    else if (TestExpress == 1) {
        runtime = 1200; /* 20 minute timeout */
    }
    else if (TestExpress == 2) {
        runtime = 600; /* 10 minute timeout */
    }
    else {
        runtime = 60; /* 1 minute timeout */
    }

    /* Calculate timeout for each test */
    test_timeout = (runtime - MARGIN) / num_tests;

    /* Add each test, passing the timeout as a parameter.
       Test 1 runs on threads created internally by the
       testing framework, while test 2 is responsible for
       managing its own threading. */
    AddTest("test1", test1, NULL, NULL, &test_timeout,
            sizeof(unsigned), ALLOW_MULTITHREAD, "Runs test 1");
    AddTest("test2", test2, NULL, NULL, &subtest_timeout,
            sizeof(unsigned), 0, "Runs test 2");

    /* Parse command line arguments, including the maximum number of
       threads to use */
    TestParseCmdLine(argc, argv);

    /* Run tests */
    PerformTests();

    /* Display test summary, if requested */
    if (GetTestSummary())
        TestSummary(stdout);

    /* If test file cleanup is not disabled, cleanup any
       test files */
    if (GetTestCleanup())
        ...;

    /* Release test infrastructure */
    TestShutdown();

    /* Shut down HDF5 */
    H5close();

    if (GetTestNumErrs() > 0 || GetTestsFailedCount() > 0)
        exit(EXIT_FAILURE);
    else
        exit(EXIT_SUCCESS);
}
\end{minted}

\end{document}