\documentclass[../HDF5_RFC.tex]{subfiles}
 
\begin{document}

% Start each appendix on a new page
\newpage

\section{Appendix: Examples}
\label{apdx:examples}

Error checking is omitted in the following examples for simplicity.

\subsection{\textbf{Example skeleton for a 'testframe' HDF5 test}}
\label{apdx:testframe_example}

\begin{minted}{C}
#include "testframe.h"

int
main(int argc, char **argv)
{
    H5open();

    /* Initialize testing framework */
    TestInit(argv[0], NULL, NULL, NULL, NULL, 0);

    /* Add each test */
    AddTest("test1", test1(), test1_setup(), test1_cleanup(),
            &test1_data, sizeof(test1_data), "Runs test 1");
    AddTest("test2", test2(), test2_setup(), test2_cleanup(),
            &test2_data, sizeof(test2_data), "Runs test 2");

    /* Parse command line arguments */
    TestParseCmdLine(argc, argv);

    /* Display testing information */
    TestInfo(stdout);

    /* Start test timer */
    TestAlarmOn();

    /* Run tests */
    PerformTests();

    /* Display test summary, if requested */
    if (GetTestSummary())
        TestSummary(stdout);

    /* Turn test timer off */
    TestAlarmOff();

    num_errs = GetTestNumErrs();

    /* Release test infrastructure */
    TestShutdown();

    if (num_errs > 0)
        exit(EXIT_FAILURE);
    else
        exit(EXIT_SUCCESS);
}
\end{minted}

\subsection{\textbf{Example skeleton for a parallel 'testframe' HDF5 test}}
\label{apdx:testframe_parallel_example}

\begin{minted}{C}
#include "testframe.h"

int
main(int argc, char **argv)
{
    int mpi_size; /* MPI communicator size */
    int mpi_rank; /* MPI process rank value */

    MPI_Init(&argc, &argv);

    MPI_Comm_rank(MPI_COMM_WORLD, &mpi_rank);
    MPI_Comm_size(MPI_COMM_WORLD, &mpi_size);

    H5open();

    /* Initialize testing framework; pass MPI rank value
       for process ID */
    TestInit(argv[0], NULL, NULL, NULL, NULL, mpi_rank);

    /* Add each test */
    AddTest("test1", test1(), test1_setup(), test1_cleanup(),
            &test1_data, sizeof(test1_data), "Runs test 1");
    AddTest("test2", test2(), test2_setup(), test2_cleanup(),
            &test2_data, sizeof(test2_data), "Runs test 2");

    /* Parse command line arguments */
    TestParseCmdLine(argc, argv);

    /* Display testing information */
    TestInfo(stdout);

    /* Start test timer */
    TestAlarmOn();

    /* Run tests */
    PerformTests();

    /* Display test summary, if requested */
    if (GetTestSummary())
        TestSummary(stdout);

    /* Turn test timer off */
    TestAlarmOff();

    /* Gather errors from all processes */
    num_errs = GetTestNumErrs();
    MPI_Allreduce(MPI_IN_PLACE, &num_errs, 1, MPI_INT,
                  MPI_MAX, MPI_COMM_WORLD);

    /* Release test infrastructure */
    TestShutdown();

    MPI_Finalize();

    if (num_errs > 0)
        exit(EXIT_FAILURE);
    else
        exit(EXIT_SUCCESS);
}
\end{minted}

\subsection{\textbf{Example skeleton for a multi-threaded HDF5 test}}
\label{apdx:multithread_example}

\begin{minted}{C}
#include "testframe.h"

static void
test1(const void *params)
{
    unsigned max_num_threads = GetTestMaxNumThreads();
    unsigned timeout = *(unsigned *)params;

    /* Calculate amount of work to do */
    ...

    /* Do work */
    ...

    return;
}

int
main(int argc, char **argv)
{
    unsigned runtime;
    unsigned num_subtests;
    unsigned subtest_timeout;
    int      TestExpress;
    int      num_errs;

    /* Initialize testing framework */
    TestInit(argv[0], NULL, NULL, NULL, NULL, 0);

    TestExpress = GetTestExpress();
    if (TestExpress == 0) {
        runtime = 0; /* Run with no timeout (runtime = 0) or pick a
                        reasonable timeout */
    }
    else if (TestExpress == 1) {
        runtime = 1800; /* 30 minute timeout */
    }
    else if (TestExpress == 2) {
        runtime = 600; /* 10 minute timeout */
    }
    else {
        runtime = 60; /* 1 minute timeout */
    }

    /* Calculate timeout for each sub-test */
    subtest_timeout = (runtime - MARGIN) / num_subtests;

    /* Add each test, passing the timeout as a parameter */
    AddTest("test1", test1(), NULL, test1_cleanup(),
            &subtest_timeout, sizeof(unsigned), "Runs test 1");
    AddTest("test2", test2(), NULL, test2_cleanup(),
            &subtest_timeout, sizeof(unsigned), "Runs test 2");

    /* Parse command line arguments, including the maximum number of
       threads to use */
    TestParseCmdLine(argc, argv);

    /* Display testing information */
    TestInfo(stdout);

    /* Start test timer; assumes refactoring of TestAlarmOn to accept
       a specified timeout. TestAlarmOn could also be refactored to
       handle this itself, given its own information about the TestExpress
       value */
    TestAlarmOn(runtime);

    /* Run sub-tests, with the test framework being responsible for spawning
       threads to run each test function. If performance becomes an issue,
       threads can be spawned once and the entire test can be encapsulated
       in a single function. Also, the testing framework may need
       infrastructure to check against the maximum allowed runtime after
       each sub-test in case a particular sub-test spills over its time
       budget. */
    PerformTests();

    /* Display test summary, if requested */
    if (GetTestSummary())
        TestSummary(stdout);

    /* Not strictly necessary to turn the alarm off, but the test alarm
       should generally be left on until right before exiting since even
       the cleanup step may be prone to hanging */
    TestAlarmOff();

    num_errs = GetTestNumErrs();

    /* Release test infrastructure */
    TestShutdown();

    if (num_errs > 0)
        exit(EXIT_FAILURE);
    else
        exit(EXIT_SUCCESS);
}
\end{minted}

\end{document}