\documentclass[../HDF5_RFC.tex]{subfiles}
 
\begin{document}

% Start each appendix on a new page
\newpage

\section{Appendix: 'Testpar' testing framework}
\label{apdx:testpar}

The following sections give an overview of HDF5's 'testpar' testing framework which is contained
within the \texttt{testpar/testpar.h} and \texttt{testpar/testpar.c} files. Details are only given
for the types, functions, etc. that appear in the relevant header files.

\subsection{Macros}

\begin{minted}{C}
#define MAINPROCESS (!mpi_rank)
\end{minted}

Convenience macro that returns a \texttt{true} value for the MPI process with rank value 0, and a
\texttt{false} value for MPI processes with rank values that aren't 0. Used to separate test logic
that should only be executed on the MPI process with rank value 0 from test logic that should be
executed on other MPI processes.

\begin{minted}{C}
#define MESG(mesg)
\end{minted}

Macro to print out a message string \texttt{mesg} as long as the test verbosity level setting is at least medium (7) and the string isn't empty.

\textbf{Note:} The specified message string will be printed by any MPI process which uses this macro.
Test programs should avoid using this macro on all MPI processes to prevent large amounts of output.

\begin{minted}{C}
#define MPI_BANNER(mesg)
\end{minted}

Macro to print out a message string \texttt{mesg} surrounded by a block of '-' characters. If the test
verbosity level setting is at least medium (7), the message string will be printed out by all MPI processes
that use the macro. Otherwise, the message string will only be printed out by the MPI process with rank
value 0.

\begin{minted}{C}
#define VRFY_IMPL(val, mesg, rankvar)
#define VRFY_G(val, mesg) VRFY_IMPL(val, mesg, mpi_rank_g)
#define VRFY(val, mesg)   VRFY_IMPL(val, mesg, mpi_rank)
#define INFO(val, mesg)
\end{minted}

Set of macros used for verifying that the condition or variable specified in \texttt{val} is \texttt{true}.

If the condition or variable specified in \texttt{val} is \texttt{true}, the message string specified in
\texttt{mesg} will be printed using the \texttt{MESG()} macro (as long as the test verbosity level setting
is at least medium (7)). Otherwise, an error message will be printed out. When the condition or variable specified in \texttt{val} is \texttt{false}, the \texttt{VRFY} macros will call \texttt{MPI\_Abort()} to
terminate the test program, as long as the test verbosity level setting is below medium (7).

The third parameter to the \texttt{VRFY\_IMPL()} macro should be the name of the variable that contains
the MPI process rank value. \texttt{VRFY()} and \texttt{VRFY\_G()} are simply convenience wrappers for
when that variable is named \texttt{mpi\_rank} or \texttt{mpi\_rank\_g}, respectively.

\begin{minted}{C}
#define SYNC(comm)
\end{minted}

Macro to perform an \texttt{MPI\_Barrier()} operation on the specified MPI Communicator. Informational
before and after messages will be printed using the \texttt{MPI\_BANNER()} macro.

\begin{minted}{C}
#define H5_PARALLEL_TEST
\end{minted}

This macro is defined primarily so that other headers included by either \texttt{testpar.h}
or the test program itself can determine whether they're being included from a parallel test
and can adjust functionality accordingly.

\begin{minted}{C}
#define FACC_DEFAULT 0x0 /* default */
#define FACC_MPIO    0x1 /* MPIO */
#define FACC_SPLIT   0x2 /* Split File */
\end{minted}

These macros are definitions for the \texttt{create\_faccess\_plist()} which are used to specify
modifications to be made to the File Access Property List that is returned from the function.

\begin{minted}{C}
#define DXFER_COLLECTIVE_IO  0x1 /* Collective IO */
#define DXFER_INDEPENDENT_IO 0x2 /* Independent IO collectively */
\end{minted}

These macros are used by various parallel tests to control test behavior depending on whether
the test is operating in collective or independent I/O mode.

\begin{minted}{C}
#define BYROW 1 /* divide into slabs of rows */
#define BYCOL 2 /* divide into blocks of columns */
#define ZROW  3 /* same as BYCOL except process 0 gets 0 rows */
#define ZCOL  4 /* same as BYCOL except process 0 gets 0 columns */
\end{minted}

These macros are used by various parallel tests to determine how to split a hyperslab selection
among the available MPI processes when performing parallel dataset I/O.

\begin{minted}{C}
#define IN_ORDER     1
#define OUT_OF_ORDER 2
\end{minted}

These macros are used by various parallel tests to determine the method for selecting points
when using a point selection for parallel dataset I/O.

\begin{minted}{C}
#define MAX_ERR_REPORT 10 /* Maximum number of errors reported */
\end{minted}

This macro is intended to put a limit on the number of errors that a parallel test program
should report in cases such as data verification loops, where several errors could be reported
at once from multiple processes.

\subsection{Enumerations}

\begin{minted}{C}
enum H5TEST_COLL_CHUNK_API {
    API_NONE = 0,
    API_LINK_HARD,
    API_MULTI_HARD,
    API_LINK_TRUE,
    API_LINK_FALSE,
    API_MULTI_COLL,
    API_MULTI_IND
};
\end{minted}

Enumeration value which some parallel tests use to determine how to perform parallel I/O

\begin{minted}{C}
typedef enum {
    IND_CONTIG,  /* Independent IO on contiguous datasets */
    COL_CONTIG,  /* Collective IO on contiguous datasets */
    IND_CHUNKED, /* Independent IO on chunked datasets */
    COL_CHUNKED  /* Collective IO on chunked datasets */
} ShapeSameTestMethods;
\end{minted}

Enumeration value which some parallel tests use to determine how to perform parallel I/O
and on what type of datasets.

\subsection{Structures}

None defined

\subsection{Global variables}

None defined

\subsection{Functions}

\subsubsection{\texttt{create\_faccess\_plist()}}

\begin{minted}{C}
hid_t
create_faccess_plist(MPI_Comm comm, MPI_Info info, int l_facc_type);
\end{minted}

Utility function to create a File Access Property List. If the macro value \texttt{FACC\_DEFAULT} is
passed for \texttt{l\_facc\_type}, the FAPL returned will be a default FAPL for serial file access. If
the value \texttt{FACC\_MPIO} is passed for \texttt{l\_facc\_type}, the FAPL will be setup for accessing
a file through the MPI I/O file driver according to the MPI Communicator and Info objects specified in \texttt{comm} and \texttt{info}. Collective metadata reads and writes will also be enabled on the FAPL.
If the value \texttt{FACC\_MPIO | FACC\_SPLIT} is passed for \texttt{l\_facc\_type}, the FAPL will be
setup for accessing a file through the multi file driver, with the file's metadata and raw data being
split into separate files. The multi file driver will use the MPI I/O driver for managing the file data
according to the MPI Communicator and Info objects specified in \texttt{comm} and \texttt{info}.
Collective metadata reads and writes will \textbf{not} be enabled on the FAPL.

\subsubsection{\texttt{point\_set()}}

\begin{minted}{C}
void
point_set(hsize_t start[], hsize_t count[], hsize_t stride[],
          hsize_t block[], size_t num_points, hsize_t coords[],
          int order);
\end{minted}

Utility function to translate a hyperslab selection, specified by the \texttt{start}, \texttt{count}, \texttt{stride} and \texttt{block} parameters, into a point selection by populating the \texttt{coords}
array. \texttt{order} may be specified as either of the macro values \texttt{IN\_ORDER} or
\texttt{OUT\_OF\_ORDER}. \texttt{IN\_ORDER} causes the \texttt{coords} array to be populated in ascending
order while iterating through the hyperslab selection's elements; \texttt{OUT\_OF\_ORDER} causes the
\texttt{coords} array to be populated in descending order, resulting in the last selected element appearing
first in the \texttt{coords} array. If \texttt{order} is specified as \texttt{OUT\_OF\_ORDER}, the
\texttt{num\_points} parameter specifies the number of elements to translate from the hyperslab selection
and should be less than or equal to the number of points that the \texttt{coords} array has allocated space for.

\end{document}