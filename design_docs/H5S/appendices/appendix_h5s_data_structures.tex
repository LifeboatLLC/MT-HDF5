\documentclass[../HDF5_RFC.tex]{subfiles}
 
\begin{document}

% Start each appendix on a new page
\newpage

\section{Appendix: H5S interface data structures}
\label{apdx:h5s_data_structures}

The following sections give an overview of the data structures in the H5S interface
and outline any concurrency issues that will have to be dealt with for each structure.
Note that initial sketches of revisions proposed in the following sections are based
off of the approach proposed in section \ref{h5s_concurrency_rcu}, so readers should
be familiar with those details first.

\subsection{\texttt{H5S\_t}}
\label{apdx:h5s_struct_h5s_t}

\begin{minted}{C}
struct H5S_t {
    H5S_extent_t extent; /* Dataspace extent (must stay first) */
    H5S_select_t select; /* Dataspace selection */
};
\end{minted}

\nameref{apdx:h5s_struct_h5s_t} is the main structure representing a dataspace object.
Concurrency issues with the structure lie within the \texttt{extent} and \texttt{select}
fields; the structure itself doesn't have any inherent concurrency issues at the top
level. Based on the initial sketch for adopting an RCU-like approach for solving H5S
concurrency issues, a revised \nameref{apdx:h5s_struct_h5s_t} structure may look like
the following:

\begin{minted}{C}
typedef struct H5S_kernel_t {
    uint32_t reader_count;
    bool     struct_replaced;
} H5S_kernel_t;

struct H5S_t {
    H5S_extent_t extent; /* Dataspace extent (must stay first) */
    H5S_select_t select; /* Dataspace selection */
    _Atomic H5S_kernel_t kernel;
};
\end{minted}

The new \texttt{H5S\_kernel\_t} structure field would essentially be used to reference count
an instance of the structure in order to determine when it can be garbage collected. The
\texttt{reader\_count} field serves as the reference count. When it drops to 0, the structure
is allowed to be reclaimed/de-allocated. The \texttt{struct\_replaced} field serves as a flag
that specifies whether the structure has actually been replaced with a new version by a writer.
If the structure hasn't been replaced (and therefore the original is still the current version),
then the structure doesn't need to be de-allocated as it is still in use. The flag would
initially be set to \texttt{false} and would be changed to \texttt{true} when a writer replaces
a version of \nameref{apdx:h5s_struct_h5s_t} with a new copy.

\subsection{\texttt{H5S\_extent\_t}}
\label{apdx:h5s_struct_h5s_extent_t}

\begin{minted}{C}
struct H5S_extent_t {
    H5O_shared_t sh_loc; /* Shared message info (must be first) */

    H5S_class_t type;    /* Type of extent */
    unsigned    version; /* Version of object header message to
                            encode this object with */
    hsize_t     nelem;   /* Number of elements in extent */

    unsigned rank; /* Number of dimensions */
    hsize_t *size; /* Current size of the dimensions */
    hsize_t *max;  /* Maximum size of the dimensions */
};
\end{minted}

\texttt{H5S\_extent\_t} is the data structure for the extent of a dataspace object.

The \texttt{sh\_loc} field can be shared among dataspace object header messages,
meaning that an investigation of HDF5's object header sharing mechanism is likely
needed to understand the effect of concurrency here.

The \texttt{version} field is set at dataspace object creation time and is only
ever modified by the function \texttt{H5S\_set\_version()}, which is only called
on copies of existing dataspaces in \texttt{H5A\_\_create()} and \texttt{H5D\_\_init\_space()}.

All remaining fields are vulnerable to concurrency issues if the extent of a dataspace
is modified concurrently. See Appendix \ref{apdx:h5s_fields_modified} for details on
when these fields might be modified.

\subsection{\texttt{H5S\_select\_t}}
\label{apdx:h5s_struct_h5s_select_t}

\begin{minted}{C}
typedef struct {
    const H5S_select_class_t *type; /* Pointer to selection's class info */

    hbool_t  offset_changed;       /* Indicate that the offset for the
                                      selection has been changed */
    hssize_t offset[H5S_MAX_RANK]; /* Offset within the extent */

    hsize_t num_elem; /* Number of elements in selection */

    union {
        H5S_pnt_list_t  *pnt_lst; /* Info about list of selected points
                                     (order is important) */
        H5S_hyper_sel_t *hslab;   /* Info about hyperslab selection */
    } sel_info;
} H5S_select_t;
\end{minted}

\texttt{H5S\_select\_t} is the data structure for the selection within a dataspace
object.

The \texttt{type} field always points to one of the 'class' structures \texttt{H5S\_sel\_all},
\texttt{H5S\_sel\_none}, \\
\texttt{H5S\_sel\_hyper} or \texttt{H5S\_select\_class\_t}, which should be unchanging. However,
this pointer can change at any time as the type of selection set within a dataspace object changes.

The \texttt{offset} and \texttt{offset\_changed} fields are primarily updated by
the \nameref{apdx:h5s_func_h5soffset_simple} function, though they are also reset to 0
and \texttt{false}, respectively, by the internal \texttt{H5S\_set\_extent\_simple()}
function.

The \texttt{num\_elem} field is updated any time that the selection in a dataspace
changes.

The \texttt{sel\_info} field contains the selection information for a dataspace
object which could end up being shared between dataspaces in the situations described
in \ref{h5s_state_sharing}. This field is only valid when the selection in the dataspace
is either a point or hyperslab selection and is updated if changing between the two,
or if the selection in the dataspace is being released.

See Appendix \ref{apdx:h5s_fields_modified} for details on when these fields might be
modified.

\subsection{\texttt{H5S\_hyper\_sel\_t}}
\label{apdx:h5s_struct_h5s_hyper_sel_t}

\begin{minted}{C}
typedef struct {
    H5S_diminfo_valid_t diminfo_valid; /* Whether the dataset has valid
                                          diminfo */

    H5S_hyper_diminfo_t diminfo;            /* Dimension info form of
                                               hyperslab selection */
    int                 unlim_dim;          /* Dimension where selection
                                               is unlimited, or -1 if none */
    hsize_t             num_elem_non_unlim; /* # of elements in a "slice"
                                               excluding the unlimited
                                               dimension */
    H5S_hyper_span_info_t *span_lst;        /* List of hyperslab span
                                               information of all
                                               dimensions */
} H5S_hyper_sel_t;
\end{minted}

\texttt{H5S\_hyper\_sel\_t} is the data structure that maintains information about a
hyperslab selection within a dataspace object. Each of the fields in this structure
can be modified as a hyperslab selection is set on a dataspace or an existing hyperslab
selection in the dataspace is refined.

\subsection{\texttt{H5S\_hyper\_span\_info\_t}}
\label{apdx:h5s_struct_h5s_hyper_span_info_t}

\begin{minted}{C}
struct H5S_hyper_span_info_t {
    unsigned count; /* Ref. count of number of spans which share this span */

    /* The following two fields define the bounding box of this set of spans
     *  and all lower dimensions, relative to the offset.
     */
    /* (NOTE: The bounds arrays are _relative_ to the depth of the span_info
     *          node in the span tree, so the top node in a 5-D span tree will
     *          use indices 0-4 in the arrays to store it's bounds information,
     *          but the next level down in the span tree will use indices 0-3.
     *          So, each level in the span tree will have the 0th index in the
     *          arrays correspond to the bounds in "this" dimension, even if
     *          it's not the highest level in the span tree.
     */
    hsize_t *low_bounds;  /* The smallest element selected in each dimension */
    hsize_t *high_bounds; /* The largest element selected in each dimension */

    /* "Operation info" fields */
    /* (Used during copies, 'adjust', 'nelem', and 'rebuild' operations) */
    /* Currently the maximum number of simultaneous operations is 2 */
    H5S_hyper_op_info_t op_info[2];

    struct H5S_hyper_span_t *head; /* Pointer to the first span of list of
                                      spans in the current dimension */
    struct H5S_hyper_span_t *tail; /* Pointer to the last span of list of
                                      spans in the current dimension */
    hsize_t                  bounds[]; /* Array for storing low & high bounds */
                                       /* (NOTE: This uses the C99 "flexible
                                          array member" feature) */
};
\end{minted}

\texttt{H5S\_hyper\_span\_info\_t} is the data structure for representing the span tree
information for a hyperslab selection.

\subsection{\texttt{H5S\_pnt\_list\_t}}
\label{apdx:h5s_struct_h5s_pnt_list_t}

\begin{minted}{C}
struct H5S_pnt_list_t {
    /* The following two fields defines the bounding box of the whole
       set of points, relative to the offset */
    hsize_t low_bounds[H5S_MAX_RANK];  /* The smallest element selected
                                          in each dimension */
    hsize_t high_bounds[H5S_MAX_RANK]; /* The largest element selected
                                          in each dimension */

    H5S_pnt_node_t *head; /* Pointer to head of point list */
    H5S_pnt_node_t *tail; /* Pointer to tail of point list */

    hsize_t last_idx; /* Index of the point after the last returned from
                         H5S__get_select_elem_pointlist() */
    H5S_pnt_node_t *last_idx_pnt; /* Point after the last returned from
                                   * H5S__get_select_elem_pointlist().
                                   * If we ever add a way to remove points
                                   * or add points in the middle of
                                   * the pointlist we will need to invalidate
                                   * these fields. */
};
\end{minted}

\texttt{H5S\_pnt\_list\_t} is the data structure that maintains information about a point
selection within a dataspace object.

The \texttt{head} and \texttt{tail} fields keep track of the head and tail of the linked
list of currently selected points. If concurrent selection of points within a dataspace
is allowed, this linked list structure should be able to be converted into a lock-free
linked list to keep close to the original design. However, keep in mind that performance
problems have previously been reported for HDF5's current linked list implementation of
point selections, so a different data structure may be warranted.

See \ref{h5s_concurrency_point} for a discussion on concurrency issues with the
\texttt{last\_idx} and \texttt{last\_idx\_pnt} fields.

The \texttt{low\_bounds} and \texttt{high\_bounds} fields are used to keep track of a bounding
box around the currently selected points in a dataspace and present an issue for concurrent
modification of point selections. If the list of points selected in a dataspace object is
concurrently modified, there needs to be some form of coordination for updating the bounding
box once all points have been selected. These fields are currently updated on a point-by-point
basis as each individual point is selected, which could potentially be done atomically on a
point-by-point basis, but may be prohibitively expensive to do so.

\subsection{\texttt{H5S\_pnt\_node\_t}}
\label{apdx:h5s_struct_h5s_pnt_node_t}

\begin{minted}{C}
struct H5S_pnt_node_t {
    struct H5S_pnt_node_t *next;  /* Pointer to next point in list */
    hsize_t                pnt[]; /* Selected point */
                                  /* (NOTE: This uses the C99 "flexible
                                     array member" feature) */
};
\end{minted}

\texttt{H5S\_pnt\_node\_t} is the data structure for a single point selected within a dataspace
object's extent. Provided that a lock-free linked list is used for the main \texttt{H5S\_pnt\_list\_t}
structure, adapting usage of this structure for concurrency should be relatively straightforward.
Otherwise, further design discussion will be needed.

\subsection{\texttt{H5S\_sel\_iter\_t}}
\label{apdx:h5s_struct_h5s_sel_iter_t}

\begin{minted}{C}
typedef struct H5S_sel_iter_t {
    /* Selection class */
    const struct H5S_sel_iter_class_t *type; /* Selection iteration
                                                class info */

    /* Information common to all iterators */
    unsigned rank;                  /* Rank of dataspace the selection
                                       iterator is operating on */
    hsize_t  dims[H5S_MAX_RANK];    /* Dimensions of dataspace the selection
                                       is operating on */
    hssize_t sel_off[H5S_MAX_RANK]; /* Selection offset in dataspace */
    hsize_t  elmt_left;             /* Number of elements left to iterate
                                       over */
    size_t   elmt_size;             /* Size of elements to iterate over */
    unsigned flags;                 /* Flags controlling iterator behavior */

    /* Information specific to each type of iterator */
    union {
        H5S_point_iter_t pnt; /* Point selection iteration information */
        H5S_hyper_iter_t hyp; /* New Hyperslab selection iteration
                                 information */
        H5S_all_iter_t   all; /* "All" selection iteration information */
    } u;
} H5S_sel_iter_t;
\end{minted}

\texttt{H5S\_sel\_iter\_t} is the data structure used for dataspace selection iterator objects.

The \texttt{type} field always points to one of the 'class' structures \texttt{H5S\_sel\_all},
\texttt{H5S\_sel\_none}, \\
\texttt{H5S\_sel\_hyper} or \texttt{H5S\_select\_class\_t}, which should be unchanging. However,
this pointer can change at any time as the dataspace object set for a selection iterator changes
via \nameref{apdx:h5s_func_h5ssel_iter_reset}.

The \texttt{rank}, \texttt{dims}, \texttt{sel\_off}, \texttt{elmt\_size} and \texttt{flags}
fields are generally unchanging after selection iterator creation. However, a concurrent call
to \nameref{apdx:h5s_func_h5ssel_iter_reset} could modify these if an iterator object is shared
between threads.

The \texttt{elmt\_left} field is updated during each call to \nameref{apdx:h5s_func_h5ssel_iter_get_seq_list}
on an iterator object. If multiple threads are calling that function on the same iterator object,
this field will be a point of contention. However, whether or not there is a reasonable use case
for multiple threads to share the same iterator object remains to be seen.

Each of the fields in the \texttt{u} union contain information that is updated during each call to
the function \nameref{apdx:h5s_func_h5ssel_iter_get_seq_list} on an iterator object. If multiple
threads are calling that function on the same iterator object, this information will be a point of
contention. However,  whether or not there is a reasonable use case for multiple threads to share
the same iterator object remains to be seen.

\end{document}