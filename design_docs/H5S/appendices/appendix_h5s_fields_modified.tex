\documentclass[../HDF5_RFC.tex]{subfiles}

\begin{document}

% Start each appendix on a new page
\newpage

\section{Appendix: Fields modified by API functions}
\label{apdx:h5s_fields_modified}

The following sections give an overview of the fields modified in a \nameref{apdx:h5s_struct_h5s_t}
structure by API functions in the H5S interface. This information is somewhat similar to that covered
in Appendix \ref{apdx:h5s_functions}, but is gathered here for now with more details and notes for
quicker review.

\subsection{'Get'-style functions}

The following functions are considered 'get'-style functions, in that their primary purpose is
to either retrieve information from a dataspace or return a new dataspace without modifying one
that may be passed in to the function. Therefore, any modifications made to a dataspace within
one of these functions presents an obvious concurrency issue to be addressed.

\subsubsection{\nameref{apdx:h5s_func_h5screate}}

\textbf{Fields modified:} None

\textbf{Notes:} No dataspace ID passed in

\subsubsection{\nameref{apdx:h5s_func_h5screate_simple}}

\textbf{Fields modified:} None

\textbf{Notes:} No dataspace ID passed in

\subsubsection{\nameref{apdx:h5s_func_h5scombine_hyperslab}}

\textbf{Fields modified:} \texttt{select.sel\_info.hslab->span\_lst}

\textbf{Notes:} This field may or may not be modified, but a much more thorough investigation is needed.
This function may internally copy the old dataspace while sharing the span tree with the new copy,
meaning that further operations in the internal call to \texttt{H5S\_select\_hyperslab()} have the
potential to modify the span tree of the original dataspace passed in. This may not actually occur, but
the surface to be audited is fairly large.

\subsubsection{\nameref{apdx:h5s_func_h5scombine_select}}

\textbf{Fields modified:} \texttt{select.sel\_info.hslab->span\_lst}

\textbf{Notes:} This function may call the internal \texttt{H5S\_combine\_hyperslab()}, so can have
similar issues as above.

\subsubsection{\nameref{apdx:h5s_func_h5sdecode}}

\textbf{Fields modified:} None

\textbf{Notes:} No dataspace ID passed in

\subsubsection{\nameref{apdx:h5s_func_h5sencode2}}

\textbf{Fields modified:} \texttt{select.sel\_info.hslab->span\_lst->op\_info[0]} \\
\texttt{select.sel\_info.hslab->diminfo\_valid} \\
\texttt{select.sel\_info.hslab->diminfo}

\textbf{Notes:} Can result in modifying the "operation info" field inside a structure while trying to
cache the results of a hyperslab operation. See section \ref{h5s_op_gen}. Can also result in modifying
the dimension info fields for a dataspace if the library needs to "rebuild" a regular selection. See
section \ref{h5s_diminfo_rebuild}.

\subsubsection{\nameref{apdx:h5s_func_h5sextent_equal}}

\textbf{Fields modified:} None

\textbf{Notes:}

\subsubsection{\nameref{apdx:h5s_func_h5sselect_intersect_block}}

\textbf{Fields modified:} \texttt{select.sel\_info.hslab->span\_lst->op\_info[0]} \\
\texttt{select.sel\_info.hslab->diminfo\_valid} \\
\texttt{select.sel\_info.hslab->diminfo}

\textbf{Notes:} Can result in modifying the "operation info" field inside a structure while trying to
cache the results of a hyperslab operation. See section \ref{h5s_op_gen}. Can also result in modifying
the dimension info fields for a dataspace if the library needs to "rebuild" a regular selection. See
section \ref{h5s_diminfo_rebuild}.

\subsubsection{\nameref{apdx:h5s_func_h5sselect_project_intersection}}

\textbf{Fields modified:} \texttt{select.sel\_info.hslab->span\_lst->op\_info[0]} \\
\texttt{select.sel\_info.hslab->diminfo\_valid} \\
\texttt{select.sel\_info.hslab->diminfo}

\textbf{Notes:} Can result in modifying the "operation info" field inside a structure while trying to
cache the results of a hyperslab operation. See section \ref{h5s_op_gen}. Can also result in modifying
the dimension info fields for a dataspace if the library needs to "rebuild" a regular selection. See
section \ref{h5s_diminfo_rebuild}.

\subsubsection{\nameref{apdx:h5s_func_h5scopy}}

\textbf{Fields modified:} None

\textbf{Notes:}

\subsubsection{\nameref{apdx:h5s_func_h5sget_regular_hyperslab}}

\textbf{Fields modified:} \texttt{select.sel\_info.hslab->diminfo\_valid} \\
\texttt{select.sel\_info.hslab->diminfo}

\textbf{Notes:} Can result in modifying the dimension info fields for a dataspace if the library needs
to "rebuild" a regular selection. See section \ref{h5s_diminfo_rebuild}.

\subsubsection{\nameref{apdx:h5s_func_h5sget_select_bounds}}

\textbf{Fields modified:} None

\textbf{Notes:}

\subsubsection{\nameref{apdx:h5s_func_h5sget_select_elem_npoints}}

\textbf{Fields modified:} None

\textbf{Notes:}

\subsubsection{\nameref{apdx:h5s_func_h5sget_select_elem_pointlist}}

\textbf{Fields modified:} \texttt{space->select.sel\_info.pnt\_lst->last\_idx} \\
\texttt{space->select.sel\_info.pnt\_lst->last\_idx\_pnt}

\textbf{Notes:} Modifies values in point selection information structure to cache the last point that
was visited and try to speed up subsequent iteration operations.

\subsubsection{\nameref{apdx:h5s_func_h5sget_select_hyper_nblocks}}

\textbf{Fields modified:} \texttt{select.sel\_info.hslab->span\_lst->op\_info[0]}

\textbf{Notes:} Can result in modifying the "operation info" field inside a structure while trying to
cache the results of a hyperslab operation. See section \ref{h5s_op_gen}.

\subsubsection{\nameref{apdx:h5s_func_h5sget_select_hyper_blocklist}}

\textbf{Fields modified:} \texttt{select.sel\_info.hslab->diminfo\_valid} \\
\texttt{select.sel\_info.hslab->diminfo}

\textbf{Notes:} Can result in modifying the dimension info fields for a dataspace if the library needs
to "rebuild" a regular selection. See section \ref{h5s_diminfo_rebuild}.

\subsubsection{\nameref{apdx:h5s_func_h5sget_select_npoints}}

\textbf{Fields modified:} None

\textbf{Notes:}

\subsubsection{\nameref{apdx:h5s_func_h5sget_select_type}}

\textbf{Fields modified:} None

\textbf{Notes:}

\subsubsection{\nameref{apdx:h5s_func_h5sget_simple_extent_ndims}}

\textbf{Fields modified:} None

\textbf{Notes:}

\subsubsection{\nameref{apdx:h5s_func_h5sget_simple_extent_dims}}

\textbf{Fields modified:} None

\textbf{Notes:}

\subsubsection{\nameref{apdx:h5s_func_h5sget_simple_extent_npoints}}

\textbf{Fields modified:} None

\textbf{Notes:}

\subsubsection{\nameref{apdx:h5s_func_h5sget_simple_extent_type}}

\textbf{Fields modified:} None

\textbf{Notes:}

\subsubsection{\nameref{apdx:h5s_func_h5sis_regular_hyperslab}}

\textbf{Fields modified:} \texttt{select.sel\_info.hslab->diminfo\_valid} \\
\texttt{select.sel\_info.hslab->diminfo}

\textbf{Notes:} Can result in modifying the dimension info fields for a dataspace if the library needs
to "rebuild" a regular selection. See section \ref{h5s_diminfo_rebuild}.

\subsubsection{\nameref{apdx:h5s_func_h5sis_simple}}

\textbf{Fields modified:} None

\textbf{Notes:}

\subsubsection{\nameref{apdx:h5s_func_h5sselect_shape_same}}

\textbf{Fields modified:} \texttt{select.sel\_info.hslab->span\_lst} \\
\texttt{select.sel\_info.hslab->diminfo\_valid} \\
\texttt{select.sel\_info.hslab->diminfo}

\textbf{Notes:} Can result in creating a span tree for the dataspace(s) if they don't already have one.
Can also result in modifying the dimension info fields for a dataspace if the library needs to "rebuild"
a regular selection. See section \ref{h5s_diminfo_rebuild}.

\subsubsection{\nameref{apdx:h5s_func_h5sselect_valid}}

\textbf{Fields modified:} None

\textbf{Notes:}

\subsection{'Set'-style functions}

The following functions are considered 'set'-style functions, in that their primary purpose is
to modify some portion, the extent or selection or both, of a dataspace.

\subsubsection{\nameref{apdx:h5s_func_h5sclose}}

\textbf{Fields modified:} \texttt{extent}, \texttt{select}

\textbf{Notes:} Frees \texttt{extent.size} and \texttt{extent.max} and sets \texttt{extent.rank} and \texttt{extent.nelem} to 0.

Performs selection-specific cleanup on \texttt{select} field. 'all', point and hyperslab selections
set \\
\texttt{select.num\_elem} to 0 ('none' selections already have this). Point selections free \\
\texttt{select.sel\_info.pnt\_lst} and set it to \texttt{NULL}. Hyperslab selections free the span
tree information in \texttt{select.sel\_info.hslab->span\_lst} then free \texttt{select.sel\_info.hslab}.

\subsubsection{\nameref{apdx:h5s_func_h5sextent_copy}}

\textbf{Fields modified:} \texttt{extent}, \texttt{select} (in destination dataspace)

\textbf{Notes:} In destination dataspace, frees \texttt{extent.size} and \texttt{extent.max}, then
re-allocates them and copies all values over from source dataspace's \texttt{extent} field. If the
selection in the destination dataspace is an 'all' selection, also updates \texttt{select.num\_elem}
in the destination dataspace.

\subsubsection{\nameref{apdx:h5s_func_h5smodify_select}}

\textbf{Fields modified:} \texttt{select} (in destination dataspace \texttt{space1\_id})

\textbf{Notes:} Depending on the path through the hyperslab code, each field in the \texttt{select}
field may be modified, other than \texttt{select.offset} and \texttt{select.offset\_changed}. Any
modifications to the union field \texttt{sel\_info} will be made to \texttt{sel\_info.hslab}.

\subsubsection{\nameref{apdx:h5s_func_h5soffset_simple}}

\textbf{Fields modified:} \texttt{select}

\textbf{Notes:} Copies the passed in offset into \texttt{select.offset} and then sets
\texttt{select.offset\_changed} to \texttt{true}.

\subsubsection{\nameref{apdx:h5s_func_h5sselect_adjust}}

\textbf{Fields modified:} \texttt{select}

\textbf{Notes:} Calls selection type-specific callback 'adjust\_s' for adjusting the selection within
a dataspace. For 'all' and 'none' selections, does nothing. For point selections, iterates through each
selected point (\texttt{select.sel\_info.pnt\_lst->head->pnt} / \texttt{...->head->next->pnt}) and adjusts
the location of each point by subtracting the passed in offset from the point coordinates. Then, updates
the bounding box of the selection in \texttt{select.sel\_info.pnt\_lst->low\_bounds} and \\
\texttt{select.sel\_info.pnt\_lst->high\_bounds}. For hyperslab selections, updates the regular selection
information in \texttt{select.sel\_info.hslab->diminfo.opt}, \\
\texttt{select.sel\_info.hslab->diminfo.low\_bounds} and \\
\texttt{select.sel\_info.hslab->diminfo.high\_bounds}, if available. Also updates the span tree information
in the nodes in \texttt{select.sel\_info.hslab->span\_lst}, if available.

\subsubsection{\nameref{apdx:h5s_func_h5sselect_all}}

\textbf{Fields modified:} \texttt{select}

\textbf{Notes:} Releases selection in dataspace by performing selection-specific cleanup on
\texttt{select} field (see entry for \nameref{apdx:h5s_func_h5sclose} in this section) and then
sets \texttt{select.type} to point to \texttt{H5S\_sel\_all} structure. Updates \texttt{select.num\_elem}
field to the number of points in the dataspace's extent.

\subsubsection{\nameref{apdx:h5s_func_h5sselect_copy}}

\textbf{Fields modified:} \texttt{select} (in destination dataspace)

\textbf{Notes:} Releases selection in destination dataspace by performing selection-specific cleanup
on \texttt{select} field (see entry for \nameref{apdx:h5s_func_h5sclose} in this section). Then, all
fields from source dataspace \texttt{select} field are copied into destination dataspace \texttt{select}
field. Finally, selection type-specific adjustments are made to \texttt{select} field of destination
dataspace. For 'all' selections, \texttt{select.num\_elem} is updated to the number of points in the
destination dataspace's extent. For 'none' selections, \texttt{select.num\_elem} is set to 0. For
point selections, the list of selected points in the source dataspace is copied into
\texttt{select.sel\_info.pnt\_lst} in the destination dataspace. For hyperslab selections, a new
instance of \nameref{apdx:h5s_struct_h5s_hyper_sel_t} is allocated and eventually assigned to
\texttt{select.sel\_info.hslab}. If the source dataspace has span tree information, a copy is made
into \texttt{select.sel\_info.hslab->span\_lst} in the destination dataspace.

\subsubsection{\nameref{apdx:h5s_func_h5sselect_elements}}

\textbf{Fields modified:} \texttt{select}

\textbf{Notes:} If current selection in dataspace isn't a point selection, or if a point selection
is being set with the \texttt{H5S\_SELECT\_SET} selection operator, releases selection in dataspace
by performing selection-specific cleanup on \texttt{select} field (see entry for
\nameref{apdx:h5s_func_h5sclose} in this section). Next, allocates a new instance of
\nameref{apdx:h5s_struct_h5s_pnt_list_t} in \texttt{select.sel\_info.pnt\_lst} if the dataspace didn't
already have a point selection within it. Then, points are added to nodes in
\texttt{select.sel\_info.pnt\_lst->head} / \\
\texttt{...->head->next}, etc. The bounding box of the selection is updated in \\
\texttt{select.sel\_info.pnt\_lst->low\_bounds} and \texttt{select.sel\_info.pnt\_lst->high\_bounds}
as points are added. Once all points have been added, \texttt{select.num\_elem} is updated according to
the points selected. Finally, \texttt{select.type} is set to point to the \texttt{H5S\_sel\_point}
structure.

\subsubsection{\nameref{apdx:h5s_func_h5sselect_hyperslab}}

\textbf{Fields modified:} \texttt{select}

\textbf{Notes:} Depending on the specific operation used, releases selection in dataspace by performing
selection-specific cleanup on \texttt{select} field (see entry for \nameref{apdx:h5s_func_h5sclose} in
this section) and then allocates a new instance of \nameref{apdx:h5s_struct_h5s_hyper_sel_t} in
\texttt{select.sel\_info.hslab}. In this case, \texttt{select.type} is changed to point to the
\texttt{H5S\_sel\_hyper} structure. Otherwise, essentially tries to combine a hyperslab selection into
the existing selection. This could result in the selection being converted into a 'none' selection
(see entry for \nameref{apdx:h5s_func_h5sselect_none} in this section). Depending on the path through
the hyperslab code, each field in the \texttt{select} field may be modified, other than
\texttt{select.offset} and \texttt{select.offset\_changed}.

\subsubsection{\nameref{apdx:h5s_func_h5sselect_none}}

\textbf{Fields modified:} \texttt{select}

\textbf{Notes:} Releases selection in dataspace by performing selection-specific cleanup on
\texttt{select} field (see entry for \nameref{apdx:h5s_func_h5sclose} in this section) and then
sets \texttt{select.type} to point to \texttt{H5S\_sel\_none} structure.

\subsubsection{\nameref{apdx:h5s_func_h5sset_extent_none}}

\textbf{Fields modified:} \texttt{extent}, \texttt{select}

\textbf{Notes:} Frees \texttt{extent.size} and \texttt{extent.max} and sets \texttt{extent.rank} and \texttt{extent.nelem} to 0. Sets \texttt{extent.type} to \texttt{H5S\_NULL}.

\subsubsection{\nameref{apdx:h5s_func_h5sset_extent_simple}}

\textbf{Fields modified:} \texttt{extent}, \texttt{select}

\textbf{Notes:} Frees \texttt{extent.size} and \texttt{extent.max} and sets \texttt{extent.rank} and \texttt{extent.nelem} to 0. Then, performs extent type-specific setup on the dataspace. For scalar
dataspaces (when the passed in \texttt{rank} parameter is 0), sets \texttt{extent.type} to
\texttt{H5S\_SCALAR} and sets \texttt{extent.nelem} to 1. For simple dataspaces, sets \texttt{extent.type}
to \texttt{H5S\_SIMPLE}, re-allocates \texttt{extent.size} and \texttt{extent.max} and copies over
passed in values. \texttt{extent.nelem} is set to the calculated number of elements.

For both scalar and simple dataspaces, \texttt{select.offset} is set to zeroes and
\texttt{select.offset\_changed} is set to \texttt{false}. If the selection in the dataspace was an
'all' selection, \texttt{select.num\_elem} is updated to reflect that all elements in the new extent
are selected.

\subsection{Other functions}

The following functions are in a separate category due to primarily operating on a different
object than a dataspace. However, a dataspace has to be passed in to some of these functions
initially, which could have secondary effects on those dataspaces.

\subsubsection{\nameref{apdx:h5s_func_h5ssel_iter_create}}

\textbf{Fields modified:} \texttt{select.sel\_info.hslab->diminfo\_valid} \\
\texttt{select.sel\_info.hslab->diminfo}

\textbf{Notes:} Dataspace selection iterators call an 'init' callback when being created which may
operate on the passed in dataspace. Can result in modifying the dimension info fields for a dataspace
if the library needs to "rebuild" a regular selection. See section \ref{h5s_diminfo_rebuild}.

\subsubsection{\nameref{apdx:h5s_func_h5ssel_iter_get_seq_list}}

\textbf{Fields modified:} None

\textbf{Notes:}

\subsubsection{\nameref{apdx:h5s_func_h5ssel_iter_reset}}

\textbf{Fields modified:} \texttt{select.sel\_info.hslab->diminfo\_valid} \\
\texttt{select.sel\_info.hslab->diminfo}

\textbf{Notes:} Dataspace selection iterators call an 'init' callback when being reset which may
operate on the passed in dataspace. Can result in modifying the dimension info fields for a dataspace
if the library needs to "rebuild" a regular selection. See section \ref{h5s_diminfo_rebuild}.

\subsubsection{\nameref{apdx:h5s_func_h5ssel_iter_close}}

\textbf{Fields modified:} None

\textbf{Notes:}

\end{document}