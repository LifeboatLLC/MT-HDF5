\documentclass[../HDF5_RFC.tex]{subfiles}
 
\begin{document}

\section{The H5S interface}
\label{h5s_interface}

HDF5's H5S interface contains functions for manipulating HDF5 dataspace objects.
These objects consist of two main components, the extent, or dimensionality, of
the dataspace object and a selection on the data elements of the dataspace within
that extent.

\subsection{Dataspace object extents}

There are currently three different types of extents that a dataspace object may
have: 'scalar', 'simple' and 'null'.

\subsubsection{'Scalar' extent}

A dataspace with a scalar extent is comprised of a single data element and is
considered to be dimensionless.

\subsubsection{'Simple' extent}

A dataspace with a simple extent is comprised of up to 32 orthogonal, evenly spaced
dimensions of data elements, with each dimension currently being able to contain up
to a \texttt{hsize\_t} (currently defined as \texttt{uint64\_t}) number of data
elements. Each dimension must be specified with an initial number of data elements
and a maximum number of data elements, which could be greater than or equal to
the initial number of elements, or \texttt{H5S\_UNLIMITED} to allow for the possibility
of that dimension growing larger than its initial size.

\subsubsection{'Null' extent}

A dataspace with a null extent contains no data elements and is considered to be
dimensionless.

\subsection{Dataspace object selections}

A selection can be made on the elements in the extent of a dataspace object to specify
which of those elements should participate in operations that the dataspace object
is used for. There are currently four different types of selections that can be made
on a dataspace object: 'all', 'none', 'point' and 'hyperslab'. Each type of selection
is implemented in a separate file in HDF5's source code by using a system of callbacks
that code in the H5S interface utilizes.

\subsubsection{'All' selection}

This selection type is set on a dataspace object with the \nameref{apdx:h5s_func_h5sselect_all}
function and signifies that all elements in the dataspace object's extent are selected
for operations involving dataspaces. This is the default selection within a newly-created
dataspace object.

\subsubsection{'None' selection}

This selection type is set on a dataspace object with the \nameref{apdx:h5s_func_h5sselect_none}
function and signifies that none of the elements in the dataspace object's extent are
selected for operations involving dataspaces.

\subsubsection{'Hyperslab' selection}

This selection type is set on a dataspace object with the \nameref{apdx:h5s_func_h5sselect_hyperslab}
function and allows selecting of a region of elements in the dataspace according to
four array parameters: \texttt{start}, \texttt{stride}, \texttt{count} and \texttt{block}.
Each of these arrays must have the same size (in terms of array elements) as the
dimensionality of the dataspace object.

The \texttt{start} array contains the offset, in each dimension, of the region of
elements to select.

The \texttt{stride} array contains the number of elements to move, in each dimension,
from the offset specified in \texttt{start} when selecting elements. For example, a
\texttt{stride} value of 1 moves to the next element in that dimension, while a
\texttt{stride} value of 2 moves to every other element in that dimension.

The \texttt{count} array contains the number of blocks (specified in the \texttt{block}
array) to select in the dataspace, in each dimension.

The \texttt{block} array specifies the size of the blocks of elements being selected,
in each dimension. For example, a block size of 1 selects blocks of single elements in
that dimension, while a block size of 2 selects blocks of two elements in that dimension.

\subsubsection{'Point' selection}

This selection type is set on a dataspace object with the \nameref{apdx:h5s_func_h5sselect_elements}
function and is used to select individual data elements in the dataspace object's extent.
Point selections are currently implemented using a simple linked list structure (see
\ref{apdx:h5s_struct_h5s_pnt_list_t}).

\end{document}